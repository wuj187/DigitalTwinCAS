\documentclass{article}

\usepackage{tabularx}
\usepackage{booktabs}

\title{Reflection Report on \progname}

\author{\authname}

\date{}

\input{Comments}
%% Common Parts

\newcommand{\progname}{Digital Twin Forest} % PUT YOUR PROGRAM NAME HERE
\newcommand{\authname}{Team \# 8, Forest Mirror
\\ Bowen Zhang
\\ Tingyu Shi
\\ Jiacheng Wu
\\ Junhong Chen
\\ Yichen Jiang} % AUTHOR NAMES                  

\usepackage{hyperref}
    \hypersetup{colorlinks=true, linkcolor=blue, citecolor=blue, filecolor=blue,
                urlcolor=blue, unicode=false}
    \urlstyle{same}
                                

\begin{document}

\maketitle

\newpage

\tableofcontents

\newpage

\section{Changes in Response to Feedback}

\subsection{SRS and Hazard Analysis}
\subsubsection{SRS changes}
Requirement Changes (including functional and non-functional changes):
\begin{itemize}
\item We removed all the requirements related to data collection. Data collection
was completed by Dr. Gonsamo's Lab.
\item  After confirming with the supervisor, our product does not need to 
run on mobile devices such as iPad or iPhones. Therefore, we removed all the 
requirements related to mobile devices.
\item We have added traceability information in functional and non-functional 
sections by adding supporting materials and dependencies.
\item We have added traceability information between requirements and use cases.
\item We removed requirements about interacting with a specific tree since 
the lab cannot provide data about a specific tree.
\item For data update, we have finalized that only users can update data 
locally.
\item We have improved rationales and the fit criteria of functional the 
non-functional requirements. 
\item We have added requirements about synchronization between forest data and 
forest models.
\end{itemize}

\noindent
Other Changes:
\begin{itemize}
\item We removed Off-the-shelf solutions since there are no other 
Digital Twin Forest for the real forest located at Turkey Point Ontario.
\item We removed all the links for pictures. All pictures have sufficient 
solutions in the current PDF file.
\item We improved grammar issues.
\item Improved Use Case Diagram.(Including behaviors about how to access 
the instruction page and update forest data.) 
\item In the ``Current Situation'' section, we added the reference for the paper.
\item We removed the coding team from the context diagram.
\item We added data hierarchy in ``Business Data Model and Data Dictionary''
section.
\item In the ``Naming Conventions and Definitions'' section, we added detailed
and correct information about the forest data. 
\end{itemize}

\subsubsection{HA Changes}
\begin{itemize}
\item We have converted all the pictures to latex code to support text search.
\item We removed the contradiction between the assumption and security 
requirements.
\item We made recommended actions consistent with the cause of the failure.
\item About the data deletion from the users, we solved this 
hazard by providing JSON files for data backup.
\end{itemize}


\subsection{Design and Design Documentation}
As mentioned above, there are new requirement added into our project. Therefore, new modules are implememted and added into the design document. The team kept the MVC arcitecture, and the models does not change. The team added several contollers for the seasonal change and tree gemeration. The reason why each type of tree has its own modules is because of the structure of the JSON files. The team stored the property of each type of tree in individual JSON files. At the same time, the design document inclides more modules than what we have in github. That is because we inclided unity built-in features such as user interface in the design document. For instance, unity has an $onClick$ function when the users intersct with a button, the team do not need to write a script for it,but the document needs to be included in the design document.
\subsection{VnV Plan and Report}
In VnV Plan, we added the Verification and Validation Plan Verification Plan section to specify the verification plan for the VnV Plan. In addition, we replaced the ambiguous wordings used in the Tests for Nonfunctional Requirements section with more specific and clear language. We corrected grammar mistakes present in the previous versions and improved the overall readability of the text. We included the links to the Development Plan, MIS, and MG in the sections where they are mentioned. We added some additional functions in the SRS document, so we added tests for these new functions. In VnV report,  we removed the empty Figure sections from the table of contents and changed some verb tenses to make the tense consistent. We also added some reports for the newly added test plans in the VnV plan. 


\section{Design Iteration (LO11)}


\noindent The iterations of our product mainly focus on two parts: the model and the method of generating the model.\\

\noindent Our modeling process can be broken down into four key phases. Initially, the team visited Turkey Point, the target forest, last semester and used LiDAR cameras to capture several 3D models of the forest. However, the scan quality was poor as they only used their iPhones and iPads. To obtain high-resolution scans, the team would require expensive cameras, so they decided to extract only the necessary parameters from the 3D scans instead of using them directly.\\

\noindent Next, we purchased some high-quality tree models from the Unity asset store, which made the completed forest look fancy. However, the precise models increased the project's size, and they did not provide the required tree variety. The scene had seven types of trees, but the asset only featured pine trees.\\

\noindent Subsequently, we attempted to download some tree models from Sketchfab, which offered numerous species with reasonable model sizes. However, most of the excellent models were not free, making it costly to buy them individually. Additionally, these models may not be supported by Unity, meaning the team could not use the Unity-built features such as the tree editor.\\

\noindent Ultimately, we returned to the Unity store and found a tree model pack that included all required tree types, with each model supporting Unity tools like terrain tool and tree editor, mentioned above. In conclusion, our project's modeling process can be summarized into four phases.\\

\noindent Our forest model generation process involves iterating three times. Initially, we attempted to duplicate the entire forest using LiDAR scanning. However, with 14 plots spanning over 10,000 square meters each, we only had access to iPads and cell phones for scanning, resulting in budget and time constraints. As a result, we moved on to phase 2 and employed manual tree planting with Unity terrain tool, experimenting with parametric modelling. This approach enabled us to control the specific tree model's location and scale while also allowing us to refine the model manually. However, the model was static, and any modifications had to be made before its completion, rendering it less flexible. Additionally, preparing 14 different preset models for each plot proved challenging and clearly is not the spatially optimized option. Therefore, we transitioned to phase 3, where we automated the tree planting process using C sharp scripts. We mathematically derived the tree distribution functions for each plot and used them with other parameters to create a forest model that closely matched our target forest.

\section{Design Decisions (LO12)}


\noindent Our starting point of this product is to deliver a method of data visualization. Therefore, we carefully designed our data hierarchy. As an illustration, we possess 14 plots and the entire forest dataset, which contains both environmental data and tree parameters.\\

\noindent Regarding the environmental data, we have compiled information on humidity, temperature, soil carbon content, soil nitrogen content, and LAI, which stands for Leaf Area Index.\\

\noindent Concerning the tree parameters, we have documented seven distinct species in our forest and their respective Diameter on Breast Height, referred to as DBH, Height, Density, and Age.\\

\noindent In addition to offering outstanding hierarchical data visualization capabilities, our product is an exciting demonstration of the non-artificial aspect of Digital Twin Earth. Our product showcases the potential for real-time model updates corresponding to current data, thus filling a gap in current digital twin technology and producing astonishing results. Different from the traditional digital twin technology, which generates the models based on the design plan of the architectures, our digital twin forest focuses on the natural environment that constantly changes. We made a huge amount of effort to provide a solution to monitor the target forest from distance. The entire model of the forest we provided in this product corresponds to the real forest, delivering a realistic and direct visualization for the stakeholders. The product provides the possibility of working with physical sensors that can be placed in the forest, which is also mentioned in section 5.4 in this document.

\noindent 


\section{Economic Considerations (LO23)}


According to \href{https://www.fortunebusinessinsights.com/digital-twin-market-106246}{a report on market size of digital twin}, we can see that the current market size is considerable. In 2020, the North America digital twin market size is 1.76 billion USD. This number increased to 2.25 billion USD in 2021. The expected market size would grow to 96.49 billion by 2029. Different from the traditional digital twin technology, our product shows another possibility of the digital twins, and make it possible to a broader view, which is Digital Twin Earth. Digital Twin Earth is a virtual replica of the Earth that can help us predict and mitigate the impacts of climate change, manage natural resources more efficiently, and optimize urban planning and infrastructure development. Our supervisor, Dr. Gonsamo, and his lab memebers are going to work based on our product for several years. And ideally, we can get a prototype in two or three years. \\

\noindent In order to create a sellable version of our product, we estimate that we will require approximately one year to place physical sensors, implement the interface, and gather sufficient data. The financial cost of the project will largely depend on the cost of the physical sensors that we will need to purchase.\\

\noindent Rather than a one-time payment, we intend to offer our product through monthly subscriptions and payment plans. Based on our analysis, we believe a fair and reasonable price would be around 20 dollars per month, considering the scientific and commercial purposes it serves. The time required to recover the cost will depend on the total cost of the project, but we estimate it to be approximately one year. Our ideal goal is to utilize the generated income to continuously improve our product, ensuring that we offer the highest quality to our users.

\section{Reflection on Project Management (LO24)}


\subsection{How Does Your Project Management Compare to Your Development Plan}

\noindent Our team's most notable accomplishment is the seamless collaboration and exceptional management we have achieved. Despite being a newly-formed group, we have exceeded expectations in terms of team management. Our meetings that happen twice a week are held consistently and have been incredibly productive, resulting in consistent progress. Furthermore, our management techniques have proven successful in engaging every member of our team, inspiring each individual to devote their full effort to the project. \\

\noindent We strictly adhered to the project management plan outlined in the Development Plan. Each task was assigned to a specific team member, and every member took on the role of team leader at least once. This workflow ensured that every team member was able to fully realize their potential. Additionally, we maintained a regular informal meeting schedule, as all team members shared the same lecture schedule. We were grateful for the opportunity to spend time together during our graduation year and would be enthusiastic about collaborating again in the future should the opportunity arise.\\

\subsection{What Went Well?}

Our project management process and technology were most effective in two areas: communication and parametric modelling. Firstly, our team excelled in communication throughout the project. We held regular team meetings and also visited Dr. Gonsamo's lab weekly. During each meeting, the team planned out the upcoming week's work and assigned reasonable workloads to each group member. Everyone participated in the discussion-making process and had a chance to share their ideas. Additionally, we presented our project progress to Dr. Gonsamo every week, conducted usability tests with the lab, and checked project requirements, ensuring a smooth and constant pace of progress.\\

Secondly, our team achieved a perfect implementation of the parametric modelling technology. Parametric modelling involves using scripts to control model parameters, as opposed to manual modification. Our team optimized the Unity terrain tool and tree editor with custom scripts, enabling us to control the coordinates, scales, and even the number of leaves for each tree. This approach boosted the flexibility of the project and reduced its overall size, as we no longer needed to save individual models for each tree. Instead, we could generate trees dynamically based on different plots. In conclusion, our team's excellent communication and implementation of parametric modelling technology were key to the success of our project management process.\\

\subsection{What Went Wrong?}


  \noindent The project encountered several key issues, namely unit testing, module guide, and time management. Firstly, the team faced challenges in conducting unit testing in Unity. Although Visual Studio supports automatic testing, the team was unable to use it when VS was linked with Unity. Furthermore, all tests had to be conducted manually within Unity due to its system architecture. Secondly, the team encountered difficulties in writing the module guide. Although around 30 scripts were written, the module hierarchy included over 60 modules due to the inclusion of Unity built-in features such as user interface and models. Defining the variables and functions of these features proved to be challenging, and the team had to rely on online manuals to explain them in natural language. Lastly, time management was a significant challenge for the project, as Digital Twin required extensive data collection and UI resource design. This left the team with limited time to work on the scripts. In summary, unit testing, module guide, and time management were the primary issues faced during the project.


\subsection{What Would you Do Differently Next Time?}


\noindent We would like to attempt to place physical sensors and automatically update data like humidity and temperature to our product. If we could have more budget and time, doing so and providing the interface would be the first thing we would do. Besides, inviting some students in Mechatronics to join us would be a great idea, for getting more ideas of cooperation between hardwares and softwares. \\

\noindent We would connect our product with official or trusted online resources. We could show more information about the broader view like climate changes. With permission, the product should automatically updates information from the online resources to local product, and the data from online resources can also perfectly join our data hierarchy. \\

\noindent If we have extra time and budget after realizing the possible modifications mentioned above, we would dedicate our time and effort on prediction. The feature of prediction would definitely involve more knowledge about environmental science. Ideally we would introduce machine learning technology as an assistance. This feature would be build based on the online resources, to realize predictions on statistics like fire or local climate change.

\end{document}