\documentclass{article}

\usepackage{tabularx}
\usepackage{booktabs}
\usepackage{graphicx}
\usepackage{amssymb}
\usepackage{color}

\title{Problem Statement and Goals\\ Digital Twin Forest}

\author{Team 8\\Jiacheng Wu, Yichen Jiang, Tingyu Shi, Bowen Zhang, Junhong Chen}

\date{\today}

\begin{document}

\maketitle

\begin{table}[hp]
\centering
\caption{Revision History} \label{TblRevisionHistory}
\begin{tabularx}{11cm}{X X X}
\toprule
\textbf{Date} & \textbf{Developer(s)} & \textbf{Change}\\
\midrule
September 16, 2022 & All team members & First version
of the problem statement and goals\\
\hline
September 17,2022 & All team members & Complete stretch
goals\\
\hline
September 26, 2022 & All team members & Modify environment 
and goals\\
\hline
March 29 ,2023 & All team members & Final Version\\
\bottomrule
\end{tabularx}
\end{table}

\section{Problem Statement}
\subsection{Problem}
A digital twin forest is a virtual representation of the 
natural forest located at Turkey Point Ontario. By taking
various data from the forest such as environmental data 
and tree parameters, we are able to model the forest with
these data. As a result, we can supervise the forest from a
distance. More specifically, we are trying to solve the 
following three problems:
\begin{itemize}
\item A virtual representation of the real forest,
allowing monitoring and analyzing from distance.
\item Visualizing important data related to scientific 
research and decision-making.
\item A forest model that can change dynamically 
according to the modification of the data.
\end{itemize}
This project can be beneficial for both commercial and scientific use. For commercial use, the product can help forest owners make decisions, and for scientific use, the product can help researchers to study climate change.

\subsection{Inputs and Outputs}

Inputs:
\begin{itemize}
    \item Real forest data collected by the lab.
\end{itemize}
Outputs:
\begin{itemize}
\item Virtual representations of the real forest. (Tree
Models)
\item Forest data visualization. 
\item Synchronisation between forest data and tree models.
\end{itemize}

\subsection{Stakeholders}
\begin{itemize}
    \item Dr. Alemu Gonsamo from School of Earth,
    Environment and Society McMaster University (Dr.
    Gonsamo is the supervisor of this project.)
    \item Forest Owners(The final project can be helpful
    for forest owners to better manage the 
    forest and make decisions)
    \item Meteorologists(The final product can be helpful 
    for researchers to study climate change)
\end{itemize}

\subsection{Environment}
\begin{itemize}
    \item 3D Scanner app: The app generates the 3D-reconstruction of the environment. It provides the team with the basic data of the forest, such as the tree heights, diameters, etc.
    \item Unity: A game engine that supports augmented reality development and model editing. The team will build the virtual forest, and design the user interface here.
    \item Virtual Studio 2019: The IDE for augmented reality implementation. It supports C Sharp auto-correction and in game tests, so it has been widely used in the field of AR development.
    \item ArcGIS: An online tool to visualize forest distribution and tree data. The team can calculate the density of different tree species and the formulas according to the tree distribution based on the satellite pictures.
\end{itemize}

\section{Goals}
\begin{itemize}
\item To create a virtual representation of the real forest, allowing for monitoring and analyzing.
\item To visualize important data related to scientific 
research and decision-making, including environmental data such as temperature and humidity, as well as tree-related data including density, height, and diameter.
\item To develop a dynamic forest model that can be modified based on the collected data.
\end{itemize}

\section{Stretch Goals}

\begin{itemize}
\item Utilize AI models to predict various values relevant to the forest, such as soil conditions and carbon dioxide levels, to enhance the study of climate change in this forest.
\item Implement seasonal changes into the virtual representation of the forest.
\end{itemize}
\end{document}