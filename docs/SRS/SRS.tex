\documentclass{article}

\usepackage{tabularx}
\usepackage{booktabs}
\usepackage{graphicx}
\usepackage{paralist}
\usepackage{listings}
\usepackage{booktabs}
\usepackage{hyperref}
\usepackage{amsfonts}
\usepackage{amsmath}
\usepackage{color}
\usepackage{fancyhdr}
\usepackage{geometry}
\usepackage{soul}
\usepackage{multirow}
\usepackage{ulem}

\title{SRS\\Digital Twin Forest}
\author{Yichen Jiang, Bowen Zhang, Jiacheng Wu, Junhong Chen, Tingyu Shi\\Team 8}

\begin{document}

\maketitle
%%%%%%%%%%%%%%%%%%%%%%%% Revision History %%%%%%%%%%%%%%%%%%%%%%%%%%%%%%%
\newpage
\begin{table}[htp]
\caption{Revision History} 
\begin{tabularx}{\textwidth}{llX}
\toprule
\textbf{Date} & \textbf{Developer(s)} & \textbf{Change}\\
\midrule
Sept 24, 2022 & All team members & Initial Document\\

\bottomrule
\end{tabularx}
\end{table}
%%%%%%%%%%%%%%%%%%%%%%%%%%%%%%%%%%%%%%%%%%%%%%%%%%%%%%%%%

\newpage

%%%%%%%%%%%% Template Name %%%%%%%%%%%%%%%%%%%%%%%%%%%%%%%
\noindent This document follows \href{https://www.cs.uic.edu/~i440/VolereMaterials/templateArchive16/c%20Volere%20template16.pdf}{\textcolor{red}{Volere Template}}.
The following are some modifications that we made to the 
original template
%%%%%%%%%%%%% Template Name End %%%%%%%%%%%%%%%%%%%%%%%%%%%%


\newpage
%%%%%%%%%%%%%%Contents%%%%%%%%%%%%%%%%
\tableofcontents
\listoftables
\listoffigures
\cleardoublepage

%%%%%%%%%%% Project Drivers %%%%%%%%%%%%
\section{Project Drivers}
\subsection{The purpose of the Project}
\subsubsection{The User Business or Background of the Project Effect}
A digital twin is a virtual representation of the real world, including physical objects, 
processes, relationships, and behaviors. Elements of a digital twin include data capture
and integration, visualization, advanced analysis including AI, automation, and information
sharing and collaboration. This project can be beneficial for two groups of users.  The first
group of users is forest owners. This project can help them to manage the forest. The 
second group of users is climate change researchers. This project can help them to study
climate change.
\subsubsection{Goals of the Project}
\begin{itemize}
    \item Implement the virtual forest, which corresponds to the target natural forest. The model of a single tree is obtained by LiDAR scanning on the field. The final project combines previous models and lab statistics to give a virtual view of the forest. 
    \item Provide basic representation of data, such as age, height, and plant density. 
\end{itemize}

\subsection{Stakeholders}
\subsubsection{The client}
\begin{itemize}
    \item Dr. Alemu Gonsamo from School of Earth, Environment and Society McMaster University. (Dr. Gonsamo is the supervisor of this project.)
    \item Dr. Spencer Smith from Computing and Software Department, McMaster University. (Dr. 
    Smith is the professor of capstone course, he will give assessments of this project.)
\end{itemize}
\subsubsection{The Customer}
\begin{itemize}
    \item Forest Owners(The final project can be helpful for forest owners to better manage the 
    forest and make decisions)
    \item Climate change researchers(The final product can be helpful for researchers to 
    study climate change)
\end{itemize}
\subsubsection{Other Stakeholders}
\begin{itemize}
    \item Dr. Gosamo's lab members(Graduate students from the lab will provide suggestions and
    data needed to assist this project)
\end{itemize}
\subsubsection{The Hands-On Users of the Product}
The hands-on users of this product are the same as customers mentioned in section 1.2.2. Users'
responsibilities are also mentioned in section 1.2.2. For subject matter experience, these 
users are master. For both forest owners and climate change researchers, they are definitely
familiar with the real forest and our product can help them better doing
their jobs. For the point of technology, we assume that they have little experience of 
\textcolor{red}{AR or VR(Comfirm Later)} technologies.
\subsubsection{Personas}
Dr. Gonsam is an assistant professor from School Earth, Environment \& Society. His research
interests include Remote Sensing of Vegetation, Phenology, Global Change Ecology, Terrestrial Carbon Cycle, Climate Change Impact. For more information, you can check his school official
website \href{https://www.science.mcmaster.ca/ees/component/comprofiler/userprofile/gonsamoa.html}{\textcolor{red}{here}}.
\subsubsection{Priorities Assigned to Users}
\begin{itemize}
    \item Key Users: Dr. Gonsamo, Forest owners, Climate change researchers
    \item Secondary Users: Dr. Smith, Lab members
\end{itemize}
\subsubsection{User Participation}
\begin{itemize}
    \item Dr. Smith: Dr. Smith will provide suggestions about project management, project 
    documents and project technologies like git.
    \item Dr. Gonsamo: Dr. Gonsamo will provide business knowledge(forest data), 
    interface prototyping and usability requirements for this project. Since Dr. Gonsamo is 
    a climate researcher, he can also provide suggestions for climate researchers.
    \item Lab members: Lab members will help this project by providing some business knowledge
    about the forest.
    \item Forest Owners: Forest owners can provide commercial data about the forest. Commercial
    data include tree cutting data, annual profits, etc.
\end{itemize}
\subsubsection{Maintenance Users and Service Technicians}
The project members will be responsible for maintaining and changing the product.
%%%%%%%%%%% Project Drivers End %%%%%%%%%%


%%%%%%%%%% Project Constraints %%%%%%%%%%%%
\section{Project Constraints}
\subsection{Mandated Constraints}
\subsubsection{Solution Constraints}
\begin{itemize}
    \item Scanning Technology: Our team will use Light Detection and Ranging(LiDAR) method
    to scan physical objects in the forest. This is a kind of Laser Scanning technology. 
    We determined to use this technology for three reasons. The first reason is that LiDAR sensors
    are accessible because they are common among Apple devices like iPhone or iPad. The 
    second reason is that LiDAR sensors can provide accurate scanning. The third reason
    is that LiDAR allows to scan large areas within a short period of time.
    \item Modeling Technology: Our team will use Unity for modeling task. Unity version 
    will be 2021.3. Our team members use different platforms(Windows or MacOS). Unity is 
    suitable for our team because it is a cross-platform software. Also, Unity has some 
    existing AR/VR tool boxes, which can speed up our modelling process.
    \item Project Documents Technology: Our team will use \LaTeX for our documents.
    \item Project Cooperation Technology: Our team will use GitHub to cooperate.
    \item Code Testing Technology: VS2019 will be used for the unit test. The team will create a unit test project (.NET Framework) that contains MSTest unit tests.
    \item Code Coverage Testing Technology: The JetBrains dotCover is a code coverage tool that integrates with VS2019. It can execute and run coverage analysis for unit tests in Visual Studio.
\end{itemize}
\subsubsection{Implementation Environment of the Current System}
Discuss the technological and physical environment in which the product is to be installed.\\
\textcolor{red}{Computer System or AR/VR(Confirm Later)}
\subsubsection{Partner or Collaborative Applications}
N/A(Our product is an independent product)
\subsubsection{Off-the-Shelf Software}
\begin{itemize}
    \item LiDAR sensors: LiDAR sensors should be used to scan the forest and collect forest data
    for modelling.
    \item Unity: Unity should be used for modelling.
\end{itemize}
\subsubsection{Anticipated Workplace Environment}
This part discusses the workplace in which the users are to work and use the product.\\
\textcolor{red}{Computer System or AR/VR(Confirm Later)}
\subsubsection{Schedule Constraints}
Please check our project schedule \href{https://github.com/wuj187/DigitalTwinCAS/tree/main/docs/DevelopmentPlan/Project_Schedule}{\textcolor{red}{here}}. The following are deadlines for demonstrations:
\begin{itemize}
    \item Proof of Concept Demonstration: 2022, Nov, 14
    \item Final Demonstration: 2023, Mar, 20
\end{itemize}
\subsubsection{Budget Constraints}
Total expenses should be exceed \$750.
\subsubsection{Enterprise Constraints}
N/A This project is not invested by any enterprise.
\subsection{Naming Conventions and Definitions}
\begin{itemize}
    \item LiDAR: Light Detection and Ranging(Scanning Technology)
    \item Plot: A square shaped area in the forest. There are 13 plots in total. 
    \item Target Forest: \textcolor{red}{Confrim Later}
\end{itemize}

\subsection{Relevant Facts and Assumptions}
\subsubsection{Relevant Facts}
The following attributes can be used to describe a forest:
\begin{itemize}
    \item Leaf Area Index:
\end{itemize}
\subsubsection{Business Rules}
\subsubsection{Assumptions}
\textbf{Fact:} The project will use AR foundation\\\\
\textbf{Assumption:}
\begin{itemize}
    \item The scanning devices are available.
    \item Team members are ready to develop and meet on time.
    \item The running device supports Unity.
\end{itemize}
%%%%%%%%%%%% Project Constraints End %%%%%%%%%


%%%%%%%%%%% Functional Requirements %%%%%%%%%%%%
\section{Functional Requirements}
\subsection{The Scope of the Work}

\subsection{Business Data Model \& Data Dictionary}

\subsection{The Scope of the Product}
Nowadays, with digitization of many objects, our world is getting dramatically easier. Many countries are focusing on simulate cities. Digitization is absolutely a new trend. This project intends to simulate a forest with data collected from forest from real world. With a successful project, people are able to anticipate the consequence by collecting data from cutting, firing and climate changes. The project can have commercial benefits such as helping farm owner make decisions and scientific benefits like studying the climate change. The project will be scanned with LiDAR in iPad pro and will be implemented with C Sharp and Unity. The project will be deployed on phones.
\subsection{Functional Requirements}
%AR版流程:用户打开手机,用户打开app,用户(首次使用)看到主页相关内容以及教程,用户点击开始,加载森林模型,模型加载完成,用户在手机中看到模型,模型位置可以固定or被用户移动,用户手持设备在森林里行走,看到的场景随镜头移动发生变化,用户点击屏幕,弹出相关数据分析,用户查看数据,用户关闭app。
%电脑版流程:用户打开软件,(首次使用)主页展示教程设置等内容,用户点击开始,加载森林模型,模型加载完成,用户在屏幕上看到完整森林模型,屏幕侧栏展示整体数据,用户通过鼠标进行模型的缩放和视角移动,用户拉近到某个plot后点击plot,展示该plot信息,用户查看信息,用户关闭程序。
\begin{enumerate}[FR1]
	\item The project must simulate models of forest based on data collected from real forest.
	\item The project must display the data of the tree which the user specifies by giving inputs.
	\item 
\end{enumerate}
%%%%%%%%%%%%% Functional Requirements End %%%%%%%%%%%



%%%%%%%%%%%%% Non-functional Requirements %%%%%%%%%%%%%
\section{Nonfunctional Requirements}
\subsection{Look and Feel Requirements}
\subsubsection{Appearance Requirements}
\begin{enumerate}
    \item[LF1.1] The product shall be easy to use.
    \item[LF1.2] The product shall have a goal that each of the 13 plots adheres to.
    \item[LF1.3] The goal of the product shall be display a lifelike forest to the users.
\end{enumerate}
\subsubsection{Style Requirements}
\begin{enumerate}[LF2.1]
    \item The product shall appear authoritative based on the real forest.
    \item The product shall appear professional.
\end{enumerate}
\subsection{Usability and Humanity Requirements}
\subsubsection{Easy of Use Requirements}
\begin{enumerate}[UH1.1]
    \item The instructions of the product shall be easy to understand.
\end{enumerate}
\subsubsection{Personalization and Internationalization Requirements}
\begin{enumerate}[UH2.1]
    \item The product shall be English only.
\end{enumerate}
\subsubsection{Learning Requirements}
\begin{enumerate}[UH3.1]
    \item The instruction of the product shall be displayed at the homepage.
    \item The users shall be able to use the product with no longer than 5-minute training. 
\end{enumerate}
\subsubsection{Understandability and Politeness Requirements}
\begin{enumerate}[UH4.1]
    \item The product shall use symbols to highlight core functions.
    \item The product shall use icons that are appealing to all ages.
\end{enumerate}
\subsubsection{Accessibility Requirements}
\begin{enumerate}[UH5.1]
    \item The product shall be usable by people who are able to tap the screen.
    \item The product's user interface should be easy to learn.
\end{enumerate}
\subsection{Performance Requirements}
\subsubsection{Speed and Latency Requirements}
\begin{enumerate}
    \item[PR1.1] The product shall respond to user actions within 1 second.
    \item[PR1.2] The product shall run at no less than 30 frames per second.
    \item[PR1.3] The product shall take no more than 5 seconds to load the models.
\end{enumerate}
\subsubsection{Safety-Critical Requirements}
N/A
\subsubsection{Precision or Accuracy Requirements}
\begin{enumerate}[PR3.1]
    \item Data displayed on the screen shall be rounded to two decimal places.
    \item The relative error of each model shall be less than 10\%.
\end{enumerate}
\subsubsection{Reliability and Availability Requirements}
\begin{enumerate}[PR4.1]
    \item The product shall be available whenever the users access to it.
    \item The product shall not crash while running.
\end{enumerate}
\subsubsection{Robustness or Fault-Tolerance Requirements}
\begin{enumerate}[PR5.1]
    \item The product shall be able to run locally.
\end{enumerate}
\subsubsection{Capacity Requirements}
\begin{enumerate}[PR6.1]
    \item The app size shall be less than 5GB.
\end{enumerate}
\subsubsection{Scalability or Extensibility Requirements}
\begin{enumerate}[PR7.1]
    \item The product shall cover all 13 plots of the forest.
    \item The functions of the product shall be modularized.
    \item New components shall be easily added to the product in the future version.
\end{enumerate}
\subsubsection{Longevity Requirements}
\begin{enumerate}[PR8.1]
    \item The product shall be expected to operate within the maximum maintenance budget for a minimum of one year. 
\end{enumerate}

\subsection{Operational and Environmental Requirements}
\subsubsection{Expected Physical Requirements}
\begin{enumerate}[OE1.1]
    \item TBD
\end{enumerate}
\subsubsection{Requirements for Interfacing with Adjacent Systems}
\begin{enumerate}[OE2.1]
    \item The product shall be used on mobile devices with photogrammetry technology.
    \item \textcolor{red}{The product shall be able to run on the majority of mobile phones with Android 10.0 or latest or IOS 11.0 or latest}.
\end{enumerate}
\subsubsection{Productization Requirements}
\begin{enumerate}[OE3.1]
    \item The product shall be distributed as an application to be installed on mobile devices.
\end{enumerate}
\subsubsection{Release Requirements}
\begin{enumerate}[OE4.1]
    \item The maintenance releases will be offered to end users weekly for at least one year.
    \item Each release shall not cause previous features to fail. 
    \item Each release shall include latest modelling. 
\end{enumerate}
\subsection{Maintainability and Support Requirements}
\subsubsection{Maintenance Requirements}
\begin{enumerate}
    \item[MS1.1] Documentation of this product shall be kept up to date.
    \item[MS1.2] All functions shall be clearly documented.
    \item[MS1.3] Any detected bug in the product shall be fixed within three days.
\end{enumerate}
\subsubsection{Supportability Requirements}
\begin{enumerate}[MS2.1]
    \item The development of the product shall collect feedback from the users to improve usability.
\end{enumerate}
\subsubsection{Adaptability Requirements}
\begin{enumerate}[MS3.1]
    \item TBD
\end{enumerate}
\subsection{Security Requirements}
\subsubsection{Access Requirements}
\begin{enumerate}[SR1.1]
    \item The product shall only be accessed by users who download the product on their mobile devices.
\end{enumerate}
\subsubsection{Integrity Requirements}
\begin{enumerate}[SR2.1]
    \item The system shall not propagate errors throughout the user's devices in case of failure.
\end{enumerate}
\subsubsection{Privacy Requirements}
\begin{enumerate}[SR3.1]
    \item The product shall not change the data of the virtual forest.
    \item The product shall not ask the users to provide personal information.
    \item The product shall not send emails to the users.
\end{enumerate}
\subsubsection{Audit Requirements}
\begin{enumerate}[SR4.1]
    \item N/A
\end{enumerate}
\subsubsection{Immunity Requirements}
\begin{enumerate}[SR5.1]
    \item N/A
\end{enumerate}
\subsection{Cultural and Political Requirements}
\subsubsection{Cultural Requirements}
\begin{enumerate}[CP1.1]
    \item The product shall not have elements that offend the users of the environment in which the system is deployed in.
\end{enumerate}
\subsubsection{Political Requirements}
\begin{enumerate}[CP2.1]
    \item The product shall not include elements which may be interpreted as a political statement.
\end{enumerate}
\subsection{Legal Requirements}
\subsubsection{Compliance Requirements}
\begin{enumerate}[LR1.1]
    \item N/A
\end{enumerate}
\subsubsection{Standards Requirements}
\begin{enumerate}[LR2.1]
    \item The product shall abide by all Canadian laws and regulations.
    \item The product shall be facing to the users of all ages.
\end{enumerate}
%%%%%%%%%%%%%%%%%%% Non-functional Requirements End %%%%%%%%%%%%

%%%%%%%%%%%%%%%%%%% Project Issues %%%%%%%%%%%%%%%%%%%%%%%%%%%%
\section{Project Issues}
\subsection{Open Issues}
The process of collecting data, building models and eventually generating a product takes a long time. During this process, real-world data may have changed due to environmental factors and human factors. Environmental factors such as thunderstorms, conflagrations, earthquakes, floods and the natural growth of plants and human factors such as cutting and planting might change the data of trees and the structure and density of the forest. When we are modelling the actual forests, The data that we use is non-real-time, which will lead to the data shown in the virtual forest that we model being different from the actual data in the real world. This might cause inaccuracy in the actual use of the product.
\subsection{Off-the-Shelf Solutions}
There are many unity tutorials online.
We get data and solution from Dr.Gonsamo's lab members.
We referred to a \href{https://reader.elsevier.com/reader/sd/pii/S1569843222001881?token=0FD852C628FAE19CABA5E197E8D7ACFF3F2161E405D1A2EC950EE68C39EE00A59ACE7E27C22E4B86F3E04611242D7160&originRegion=us-east-1&originCreation=20220925224650}{paper}  to design our project. 
\subsection{New Problems}
TBD
\subsection{Tasks}
\subsubsection{Project Planning}
The project is planned to be done before February 2, 2023.
\subsubsection{Planning of the Development Phases}
There are five development phases of this project:
\begin{itemize}
    \item design feature
    \item measure
    \item implement
    \item test
    \item apply
\end{itemize}
\subsection{Migration to the new Product}
N/A
\subsection{Risks}
\begin{enumerate}
    \item Bad weather when field modelling may result in inadequate modelling.
    \item Budget might not cover the cost.
    \item Excessive schedule pressure.
    \item The mobile devices may not provide sufficient computing capacities to test the models.
    \item The trees might be too high to scan all aspects of the trees or result in a poor precision.
    \item The project might occupy too much memory of the devices.
    \item The natural forest continuously changes, which brings high maintenance cost to update related information. 
\end{enumerate}
\subsection{Costs}
The cost of this project will not exceed 750 Canadian dollars.
\subsection{User Documentation and Training}
User manuals with a few lines of instructions. 
\subsection{Waiting Room}
\begin{enumerate}
    \item The product shall represent the overall data of forests, such as amount of logging, the situation of growth, etc.
    \item The product shall record significant data for later use.
\end{enumerate}
\subsection{Ideas for Solutions}
The representation of overall data of the forest might be realized in a separate module. And same for the recorded significant data. Our current project is designed to run locally on a certain device, while a possible solution could be, to design an online mode for the users to access to the latest information. The users might click a certain button to browse the overall data of the forest and its historical versions. 

%%%%%%%%%%%%%%%%% Project Issues End %%%%%%%%%%%%%%%%%%%

\end{document}