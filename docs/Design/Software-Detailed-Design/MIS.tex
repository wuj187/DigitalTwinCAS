

\documentclass[12pt, titlepage]{article}

\usepackage{amsmath, mathtools}

\usepackage[round]{natbib}
\usepackage{amsfonts}
\usepackage{amssymb}
\usepackage{graphicx}
\usepackage{colortbl}
\usepackage{xr}
\usepackage{hyperref}
\usepackage{longtable}
\usepackage{xfrac}
\usepackage{tabularx}
\usepackage{float}
\usepackage{siunitx}
\usepackage{booktabs}
\usepackage{multirow}
\usepackage[section]{placeins}
\usepackage{caption}
\usepackage{fullpage}

\hypersetup{
bookmarks=true,     % show bookmarks bar?
colorlinks=true,       % false: boxed links; true: colored links
linkcolor=red,          % color of internal links (change box color with linkbordercolor)
citecolor=blue,      % color of links to bibliography
filecolor=magenta,  % color of file links
urlcolor=cyan          % color of external links
}

\usepackage{array}

\input{Comments}
%% Common Parts

\newcommand{\progname}{Digital Twin Forest} % PUT YOUR PROGRAM NAME HERE
\newcommand{\authname}{Team \# 8, Forest Mirror
\\ Bowen Zhang
\\ Tingyu Shi
\\ Jiacheng Wu
\\ Junhong Chen
\\ Yichen Jiang} % AUTHOR NAMES                  

\usepackage{hyperref}
    \hypersetup{colorlinks=true, linkcolor=blue, citecolor=blue, filecolor=blue,
                urlcolor=blue, unicode=false}
    \urlstyle{same}
                                

%%%%%%%%%% new command %%%%%%%%%%%%%%
\newcounter{acnum}
\newcommand{\actheacnum}{AC\theacnum}
\newcommand{\acref}[1]{AC\ref{#1}}

\newcounter{ucnum}
\newcommand{\uctheucnum}{UC\theucnum}
\newcommand{\uref}[1]{UC\ref{#1}}

\newcounter{mnum}
\newcommand{\mthemnum}{M\themnum}
\newcommand{\mref}[1]{M\ref{#1}}
%%%%%%%%%% new command end %%%%%%%%%%%%

\begin{document}

\title{Module Interface Specification for \progname{}}

\author{\authname}

\date{\today}

\maketitle

\pagenumbering{roman}

\section{Revision History}

\begin{tabularx}{\textwidth}{p{3cm}p{2cm}X}
\toprule {\bf Date} & {\bf Version} & {\bf Notes}\\
\midrule
Jan 14 & 1.0 & First Version\\
\bottomrule
\end{tabularx}

~\newpage

\section{Symbols, Abbreviations and Acronyms}

See SRS Documentation at \href{https://github.com/wuj187/DigitalTwinCAS/blob/main/docs/DocRevision/SRSRevision/SRSRevision.pdf}{here}.
\\

\begin{tabular}{l l} 
  \toprule		
  \textbf{symbol} & \textbf{description}\\
  \midrule 
  AC & Anticipated Change\\
  DAG & Directed Acyclic Graph \\
  M & Module \\
  MG & Module Guide \\
  OS & Operating System \\
  R & Requirement\\
  FR & Functional Requirement\\
  NFR & Non-Functional Requirement\\
  SC & Scientific Computing \\
  SRS & Software Requirements Specification\\
  \progname & Explanation of program name\\
  UC & Unlikely Change \\
  MVC & Model, Viewer, Controller\\
  GUI & Graphical User Interface\\
  LAI & Leaf Area Index\\
  DBH & Diameter at breast height\\
  \bottomrule
\end{tabular}\\
\newpage

\tableofcontents

\newpage

\pagenumbering{arabic}

\section{Introduction}

The following document details the Module Interface 
Specifications for \progname{}. A digital twin is a virtual 
representation of the real world, including physical objects, 
processes, relationships, and behaviors. Elements of a 
digital twin include data capture
and integration, visualization, and advanced analysis 
including AI, automation, and information sharing and 
collaboration. This project can be beneficial for two groups 
of users.  The first group of users is forest owners. This 
project can help them to manage the forest. The second group 
of users is meteorologists. This project can help them to 
do research. 

\noindent Complementary documents include the System 
Requirement Specifications and Module Guide.  The full 
documentation and implementation can be
found at \href{https://github.com/wuj187/DigitalTwinCAS}
{here}. 

\section{Notation}

\noindent The structure of the MIS for modules comes from \citet{HoffmanAndStrooper1995},
with the addition that template modules have been adapted from
\cite{GhezziEtAl2003}.  The mathematical notation comes from 
Chapter 3 of
\citet{HoffmanAndStrooper1995}.  For instance, the symbol := 
is used for a multiple assignment statement and conditional 
rules follow the form $(c_1 \Rightarrow r_1 | c_2 \Rightarrow
r_2 | ... | c_n \Rightarrow r_n )$.

The following table summarizes the primitive data types used by \progname. 

\begin{center}
\renewcommand{\arraystretch}{1.2}
\noindent 
\begin{tabular}{l l p{7.5cm}} 
\toprule 
\textbf{Data Type} & \textbf{Notation} & \textbf{Description}\\ 
\midrule
character & char & a single symbol or digit\\
integer & $\mathbb{Z}$ & a number without a fractional component in (-$\infty$, $\infty$) \\
natural number & $\mathbb{N}$ & a number without a fractional component in [1, $\infty$) \\
real & $\mathbb{R}$ & any number in (-$\infty$, $\infty$)\\
Boolean & Boolean & a value that takes either True or False\\
\bottomrule
\end{tabular} 
\end{center}

\noindent
The specification of \progname \ uses some derived data types: sequences, strings, and
tuples. Sequences are lists filled with elements of the same data type. Strings
are sequences of characters. Tuples contain a list of values, potentially of
different types. In addition, \progname \ uses functions, which
are defined by the data types of their inputs and outputs. Local functions are
described by giving their type signature followed by their specification.

\newpage

\section{Module Decomposition}

\begin{table}[H]
\caption{Module Hierarchy(Models)}
\label{TblModels}

\centering
\begin{tabular}{p{0.3\textwidth} p{0.6\textwidth}}
\toprule
\textbf{Level 1} & \textbf{Level 2}\\
\midrule

\multirow{14}{0.3\textwidth}{Model Modules}
& \refstepcounter{mnum} \mthemnum \label{Model1}: ForestTrees \\
& \refstepcounter{mnum} \mthemnum \label{Model2}: ForestSky  \\
& \refstepcounter{mnum} \mthemnum \label{Model3}: ForestTerrain \\
& \refstepcounter{mnum} \mthemnum \label{Model4}: RedPine  \\
& \refstepcounter{mnum} \mthemnum \label{Model5}: Oak   \\
& \refstepcounter{mnum} \mthemnum \label{Model6}: Beech  \\
& \refstepcounter{mnum} \mthemnum \label{Model7}: Birch  \\ 
& \refstepcounter{mnum} \mthemnum \label{Model8}: WhitePine  \\
& \refstepcounter{mnum} \mthemnum \label{Model9}: RedMaple \\
& \refstepcounter{mnum} \mthemnum \label{Model10}: RedOak  \\
& \refstepcounter{mnum} \mthemnum \label{Model11}: EnvData  \\
& \refstepcounter{mnum} \mthemnum \label{Model12}: PlotData  \\
& \refstepcounter{mnum} \mthemnum \label{Model13}: FirstPersonPlayer  \\
& \refstepcounter{mnum} \mthemnum \label{Model14}: JsonFile  \\
\bottomrule

\end{tabular}
\end{table}

\newpage

\begin{table}[H]
\caption{Module Hierarchy(First Viewers Table)}
\label{TblViewers1}

\centering
\begin{tabular}{p{0.3\textwidth} p{0.6\textwidth}}
\toprule
\textbf{Level 1} & \textbf{Level 2}\\
\midrule

\multirow{13}{0.3\textwidth}{Viewer Modules}
& \refstepcounter{mnum} \mthemnum \label{Viwer1}: MainPageDisplay \\
& \refstepcounter{mnum} \mthemnum \label{Viwer2}: StartButton \\
& \refstepcounter{mnum} \mthemnum \label{Viwer3}: InstructionButton \\
& \refstepcounter{mnum} \mthemnum \label{Viwer4}: ContactUsButton \\
& \refstepcounter{mnum} \mthemnum \label{Viwer5}: QuitButton \\
& \refstepcounter{mnum} \mthemnum \label{Viwer6}: InstructionInfoDisplay \\
& \refstepcounter{mnum} \mthemnum \label{Viwer7}: ContactUsInfoDisplay \\ 
& \refstepcounter{mnum} \mthemnum \label{Viwer8}: BackButton \\
& \refstepcounter{mnum} \mthemnum \label{Viwer9}: UpdateDataDisplay \\
& \refstepcounter{mnum} \mthemnum \label{Viwer10}: EnvDataSelectionButton \\
& \refstepcounter{mnum} \mthemnum \label{Viwer11}: DataTypeSelectionButtons \\
& \refstepcounter{mnum} \mthemnum \label{Viwer12}: NewDataInputBox \\
& \refstepcounter{mnum} \mthemnum \label{Viwer13}: SaveButton \\
\bottomrule
\end{tabular}
\end{table}

\newpage


\begin{table}[H]
\caption{Module Hierarchy(Second Viewers Table)}
\label{TblViewers2}

\centering
\begin{tabular}{p{0.3\textwidth} p{0.6\textwidth}}
\toprule
\textbf{Level 1} & \textbf{Level 2}\\
\midrule

\multirow{10}{0.3\textwidth}{Viewer Modules}
& \refstepcounter{mnum} \mthemnum \label{Viwer14}: CurrentDataDisplay \\
& \refstepcounter{mnum} \mthemnum \label{Viwer15}: PlotSelectionDropDown \\
& \refstepcounter{mnum} \mthemnum \label{Viwer16}: TreeTypeSelectionDropDown \\
& \refstepcounter{mnum} \mthemnum \label{Viwer17}: UpdateDataButton \\
& \refstepcounter{mnum} \mthemnum \label{Viwer18}: ForestDisplay \\
& \refstepcounter{mnum} \mthemnum \label{Viwer19}: ShowEnvDataButton \\
& \refstepcounter{mnum} \mthemnum \label{Viwer20}: ShowTreeParamButton \\
& \refstepcounter{mnum} \mthemnum \label{Viwer21}: EnvDataDisplay \\
& \refstepcounter{mnum} \mthemnum \label{Viwer22}: TreeParamDisplay \\
& \refstepcounter{mnum} \mthemnum \label{Viwer23}: PauseIndicatorDisplay \\
\bottomrule
\end{tabular}
\end{table}

\newpage

\begin{table}[H]
\caption{Module Hierarchy(Controllers)}
\label{TblControllers}

\centering
\begin{tabular}{p{0.3\textwidth} p{0.6\textwidth}}
\toprule
\textbf{Level 1} & \textbf{Level 2}\\
\midrule

\multirow{18}{0.3\textwidth}{Controller Modules}
& \refstepcounter{mnum} \mthemnum \label{Controller1}: JsonFileReader \\
& \refstepcounter{mnum} \mthemnum \label{Controller2}: JsonFileWriter  \\
& \refstepcounter{mnum} \mthemnum \label{Controller3}: PauseManager  \\
& \refstepcounter{mnum} \mthemnum \label{Controller4}: PlayerMovement  \\
& \refstepcounter{mnum} \mthemnum \label{Controller5}: NewDataInputBoxController\\
& \refstepcounter{mnum} \mthemnum \label{Controller6}: StartButtonController  \\
& \refstepcounter{mnum} \mthemnum \label{Controller7}: InstructionButtonController \\
& \refstepcounter{mnum} \mthemnum \label{Controller8}: UpdateDataButtonController\\
& \refstepcounter{mnum} \mthemnum \label{Controller9}: ContactUsButtonController \\
& \refstepcounter{mnum} \mthemnum \label{Controller10}: QuitButtonController  \\
& \refstepcounter{mnum} \mthemnum \label{Controller11}: BackButtonController \\
& \refstepcounter{mnum} \mthemnum \label{Controller12}: PlotSelectionDropDownController \\
& \refstepcounter{mnum} \mthemnum \label{Controller13}: TreeTypeSelectionDropDownController \\
& \refstepcounter{mnum} \mthemnum \label{Controller14}: ShowEnvDataButtonController \\
& \refstepcounter{mnum} \mthemnum \label{Controller15}: ShowTreeParamButtonController \\
& \refstepcounter{mnum} \mthemnum \label{Controller16}: EnvDataSelectionButtonController\\
& \refstepcounter{mnum} \mthemnum \label{Controller17}: DataTypeSelectionButtonsController \\
& \refstepcounter{mnum} \mthemnum \label{Controller18}: SaveButtonController \\
\bottomrule

\end{tabular}
\end{table}

\newpage

%%%%%%%%%%% Forest Trees starts %%%%%%%%%%%%%%%%%%
\section{MIS of Forest Trees (\mref{Model1})} 

\subsection{Module}

ForestTrees

\subsection{Uses}

UnityPlaceTreeWizard

\subsection{Syntax}

\subsubsection{Exported Constants}
None
\subsubsection{Exported Access Programs}

\begin{center}
\begin{tabular}{|l| l | l | p{5cm}|}
\hline
\textbf{Name} & \textbf{In} & \textbf{Out} & \textbf{Exceptions} \\
\hline
GenerateTree & s: Int; t: Double & TreeModels & IllegalArgumentException \\
\hline
DeleteTree & s: Int &  &  \\
\hline
\end{tabular}
\end{center}

\subsection{Semantics}

\subsubsection{State Variables}
None

\subsubsection{Environment Variables}

TreeModdel: the asset bundle of different types of tree models.
Brush: the built-in brush to erase the trees.

\subsubsection{Assumptions}

The input parameters will match the given specification.

\subsubsection{Access Routine Semantics}

\noindent GenerateTree(s, t):
\begin{itemize}
\item transition: Unity generates tree models randomly based on the given number and tree height.
\item output: None
\item exception: None
\end{itemize}

\noindent DeleteTree(s):
\begin{itemize}
\item transition: Delete trees by clicking on the brush and erase the workspace.
\item output: None
\item exception: None
\end{itemize}

\subsubsection{Local Functions}
None
%%%%%%%%%%%%%%%%%%%%%%%%% Forest Trees ends %%%%%%%%%%%%%

\newpage

%%%%%%%%%%% Forest Sky starts %%%%%%%%%%%%%%%%%%
\section{MIS of Forest Sky (\mref{Model2})} 
\subsection{Module}

SkyBox

\subsection{Uses}

UnityLightning

\subsection{Syntax}

\subsubsection{Exported Constants}
None
\subsubsection{Exported Access Programs}

\begin{center}
\begin{tabular}{|l| l | l | p{5cm}|}
\hline
\textbf{Name} & \textbf{In} & \textbf{Out} & \textbf{Exceptions} \\
\hline
SetSkyBox & s: Texture &  &  \\
\hline
\end{tabular}
\end{center}

\subsection{Semantics}

\subsubsection{State Variables}
None

\subsubsection{Environment Variables}

SkyTexture: imported picture of the skybox.

\subsubsection{Assumptions}

Unity only takes valid texture file type as input.

\subsubsection{Access Routine Semantics}

\noindent SetSkybox(s):
\begin{itemize}
\item transition: set the current skybox to the selected texture file.
\item output: None
\item exception: None
\end{itemize}

\subsubsection{Local Functions}
None
%%%%%%%%%%%%%%%%%%%%%%%%% Forest Sky ends %%%%%%%%%%%%%

\newpage

%%%%%%%%%%% Forest Terrain starts %%%%%%%%%%%%%%%%%%
\section{MIS of Forest Terrain (\mref{Model3})} 
\subsection{Module}

ForestTerrain

\subsection{Uses}

UnityTerrain

\subsection{Syntax}

\subsubsection{Exported Constants}
None
\subsubsection{Exported Access Programs}

\begin{center}
\begin{tabular}{|l| l | l | p{5cm}|}
\hline
\textbf{Name} & \textbf{In} & \textbf{Out} & \textbf{Exceptions} \\
\hline
SetHeight &  &  &  \\
\hline
\end{tabular}
\end{center}

\subsection{Semantics}

\subsubsection{State Variables}
None

\subsubsection{Environment Variables}

Brush: brushes to set the shape and height of the terrain.

\subsubsection{Assumptions}

None

\subsubsection{Access Routine Semantics}

\noindent SetHeight():
\begin{itemize}
\item transition: Change the height of the current terrain with different Unity terrain brushes.
\item output: None
\item exception: None
\end{itemize}

\subsubsection{Local Functions}
None
%%%%%%%%%%%%%%%%%%%%%%%%% Forest Terrain ends %%%%%%%%%%%%%

\newpage

%%%%%%%%%%%%%%%%%%% RedPine starts %%%%%%%%%%%
\newcommand{\tn}{Red Pine }
\newcommand{\tmn}{RedPine}
\newcommand{\constn}{Red\ Pine}

\section{MIS of \tn (\mref{Model4})}

\subsection{Module}
\tmn

\subsection{Uses}
None

\subsection{Syntax}
\subsubsection{Exported Constants}
None
\subsubsection{Exported Access Programs}

\begin{center}
\begin{tabular}{|l|l|l| p{5cm}|}
\hline
\textbf{Name} & \textbf{In} & \textbf{Out} & \textbf{Exceptions} \\
\hline
new \tmn & & \tmn & \\
\hline
getTreeName & &String & \\
\hline 
setDensity & String & & IllegalArgumentException\\
\hline
getDensity & & String & \\
\hline
setDBH & String & & IllegalArgumentException\\
\hline
getDBH & & String & \\
\hline
setHeight & String & & IllegalArgumentException\\
\hline
getHeight & & String & \\
\hline
setAge & String & & IllegalArgumentException\\
\hline
getAge & & String & \\
\hline
\end{tabular}
\end{center}

\subsection{Semantics}

\subsubsection{State Variables}
$\mathit{Treename: String}$\\
$\mathit{Density: String}$\\
$\mathit{DBH: String}$\\
$\mathit{Height: String}$\\
$\mathit{Age: String}$\\

\subsubsection{Environment Variables}
None

\subsubsection{Assumptions}
None

\subsubsection{Access Routine Semantics}

\noindent new \tmn():
\begin{itemize}
\item transition: $\mathit{Treename, Density, DBH, Height,
Age := ``\constn", ``", ``", ``", ``"}$
\item output: $\mathit{out := self}$
\item exception: None 
\end{itemize}

\noindent getTreeName():
\begin{itemize}
\item transition: None
\item output: $\mathit{out := Treename}$
\item exception: None
\end{itemize}

\noindent setDensity(newDensity):
\begin{itemize}
\item transition: $\mathit{Density := newDensity}$
\item output: None
\item exception: $\neg$ isValidString(newDensity) $\implies$
IllegalArgumentException
\end{itemize}

\noindent getDensity():
\begin{itemize}
\item transition: None
\item output: $\mathit{out := Density}$
\item exception: None
\end{itemize}

\noindent setDBH(newDBH):
\begin{itemize}
\item transition: $\mathit{DBH := newDBH}$
\item output: None
\item exception: $\neg$ isValidString(newDBH) $\implies$
IllegalArgumentException
\end{itemize}
\newpage
\noindent getDBH():
\begin{itemize}
\item transition: None
\item output: $\mathit{out := DBH}$
\item exception: None
\end{itemize}

\noindent setHeight(newHeight):
\begin{itemize}
\item transition: $\mathit{Height := newHeight}$
\item output: None
\item exception: $\neg$ isValidString(newHeight) $\implies$
IllegalArgumentException
\end{itemize}

\noindent getHeight():
\begin{itemize}
\item transition: None
\item output: $\mathit{out := Height}$
\item exception: None
\end{itemize}

\noindent setAge(newAge):
\begin{itemize}
\item transition: $\mathit{Age := newAge}$
\item output: None
\item exception: $\neg$ isValidString(newAge) $\implies$
IllegalArgumentException
\end{itemize}

\noindent getAge():
\begin{itemize}
\item transition: None
\item output: $\mathit{out := Age}$
\item exception: None
\end{itemize}

\subsubsection{Local Functions}
ValidCharacters = \{``1", ``2", ``3", ``4", ``5", ``6", ``7"
, ``8", ``9", ``0", ``."\}\\

\noindent isValidString(S): String $\rightarrow$ $\mathbb{B}$ \\

\noindent isValidString(S) = $\forall(i : \mathbb{Z} | 0 \leq
i < |S| : S[i] \in $ ValidCharacters)
 %%%%%%%%%%%%%%%%%%%%% Red Pine ends %%%%%%%%%%%%%

\newpage

%%%%%%%%%%%%%%%%%%% Oak starts %%%%%%%%%%%
\renewcommand{\tn}{Oak }
\renewcommand{\tmn}{Oak}
\renewcommand{\constn}{Oak}

\section{MIS of \tn (\mref{Model5})}

\subsection{Module}
\tmn

\subsection{Uses}
None

\subsection{Syntax}
\subsubsection{Exported Constants}
None
\subsubsection{Exported Access Programs}

\begin{center}
\begin{tabular}{|l|l|l| p{5cm}|}
\hline
\textbf{Name} & \textbf{In} & \textbf{Out} & \textbf{Exceptions} \\
\hline
new \tmn & & \tmn & \\
\hline
getTreeName & &String & \\
\hline 
setDensity & String & & IllegalArgumentException\\
\hline
getDensity & & String & \\
\hline
setDBH & String & & IllegalArgumentException\\
\hline
getDBH & & String & \\
\hline
setHeight & String & & IllegalArgumentException\\
\hline
getHeight & & String & \\
\hline
setAge & String & & IllegalArgumentException\\
\hline
getAge & & String & \\
\hline
\end{tabular}
\end{center}

\subsection{Semantics}

\subsubsection{State Variables}
$\mathit{Treename: String}$\\
$\mathit{Density: String}$\\
$\mathit{DBH: String}$\\
$\mathit{Height: String}$\\
$\mathit{Age: String}$\\

\subsubsection{Environment Variables}
None

\subsubsection{Assumptions}
None

\subsubsection{Access Routine Semantics}

\noindent new \tmn():
\begin{itemize}
\item transition: $\mathit{Treename, Density, DBH, Height,
Age := ``\constn", ``", ``", ``", ``"}$
\item output: $\mathit{out := self}$
\item exception: None 
\end{itemize}

getTreeName():
\begin{itemize}
\item transition: None
\item output: $\mathit{out := Treename}$
\item exception: None
\end{itemize}

\noindent setDensity(newDensity):
\begin{itemize}
\item transition: $\mathit{Density := newDensity}$
\item output: None
\item exception: $\neg$ isValidString(newDensity) $\implies$
IllegalArgumentException
\end{itemize}

\noindent getDensity():
\begin{itemize}
\item transition: None
\item output: $\mathit{out := Density}$
\item exception: None
\end{itemize}

\noindent setDBH(newDBH):
\begin{itemize}
\item transition: $\mathit{DBH := newDBH}$
\item output: None
\item exception: $\neg$ isValidString(newDBH) $\implies$
IllegalArgumentException
\end{itemize}
\newpage
\noindent getDBH():
\begin{itemize}
\item transition: None
\item output: $\mathit{out := DBH}$
\item exception: None
\end{itemize}

\noindent setHeight(newHeight):
\begin{itemize}
\item transition: $\mathit{Height := newHeight}$
\item output: None
\item exception: $\neg$ isValidString(newHeight) $\implies$
IllegalArgumentException
\end{itemize}

\noindent getHeight():
\begin{itemize}
\item transition: None
\item output: $\mathit{out := Height}$
\item exception: None
\end{itemize}

\noindent setAge(newAge):
\begin{itemize}
\item transition: $\mathit{Age := newAge}$
\item output: None
\item exception: $\neg$ isValidString(newAge) $\implies$
IllegalArgumentException
\end{itemize}

\noindent getAge():
\begin{itemize}
\item transition: None
\item output: $\mathit{out := Age}$
\item exception: None
\end{itemize}

\subsubsection{Local Functions}
ValidCharacters = \{``1", ``2", ``3", ``4", ``5", ``6", ``7"
, ``8", ``9", ``0", ``."\}\\

\noindent isValidString(S): String $\rightarrow$ $\mathbb{B}$ \\

\noindent isValidString(S) = $\forall(i : \mathbb{Z} | 0 \leq
i < |S| : S[i] \in $ ValidCharacters)
 %%%%%%%%%%%%%%%%%%%%% Oak ends %%%%%%%%%%%%%

\newpage

%%%%%%%%%%%%%%%%%%% Beeck starts %%%%%%%%%%%
\renewcommand{\tn}{Beech }
\renewcommand{\tmn}{Beech}
\renewcommand{\constn}{Beech}

\section{MIS of \tn (\mref{Model6})}

\subsection{Module}
\tmn

\subsection{Uses}
None

\subsection{Syntax}
\subsubsection{Exported Constants}
None
\subsubsection{Exported Access Programs}

\begin{center}
\begin{tabular}{|l|l|l| p{5cm}|}
\hline
\textbf{Name} & \textbf{In} & \textbf{Out} & \textbf{Exceptions} \\
\hline
new \tmn & & \tmn & \\
\hline
getTreeName & &String & \\
\hline 
setDensity & String & & IllegalArgumentException\\
\hline
getDensity & & String & \\
\hline
setDBH & String & & IllegalArgumentException\\
\hline
getDBH & & String & \\
\hline
setHeight & String & & IllegalArgumentException\\
\hline
getHeight & & String & \\
\hline
setAge & String & & IllegalArgumentException\\
\hline
getAge & & String & \\
\hline
\end{tabular}
\end{center}

\subsection{Semantics}

\subsubsection{State Variables}
$\mathit{Treename: String}$\\
$\mathit{Density: String}$\\
$\mathit{DBH: String}$\\
$\mathit{Height: String}$\\
$\mathit{Age: String}$\\

\subsubsection{Environment Variables}
None

\subsubsection{Assumptions}
None

\subsubsection{Access Routine Semantics}

\noindent new \tmn():
\begin{itemize}
\item transition: $\mathit{Treename, Density, DBH, Height,
Age := ``\constn", ``", ``", ``", ``"}$
\item output: $\mathit{out := self}$
\item exception: None 
\end{itemize}

\noindent getTreeName():
\begin{itemize}
\item transition: None
\item output: $\mathit{out := Treename}$
\item exception: None
\end{itemize}

\noindent setDensity(newDensity):
\begin{itemize}
\item transition: $\mathit{Density := newDensity}$
\item output: None
\item exception: $\neg$ isValidString(newDensity) $\implies$
IllegalArgumentException
\end{itemize}

\noindent getDensity():
\begin{itemize}
\item transition: None
\item output: $\mathit{out := Density}$
\item exception: None
\end{itemize}

\noindent setDBH(newDBH):
\begin{itemize}
\item transition: $\mathit{DBH := newDBH}$
\item output: None
\item exception: $\neg$ isValidString(newDBH) $\implies$
IllegalArgumentException
\end{itemize}
\newpage
\noindent getDBH():
\begin{itemize}
\item transition: None
\item output: $\mathit{out := DBH}$
\item exception: None
\end{itemize}

\noindent setHeight(newHeight):
\begin{itemize}
\item transition: $\mathit{Height := newHeight}$
\item output: None
\item exception: $\neg$ isValidString(newHeight) $\implies$
IllegalArgumentException
\end{itemize}

\noindent getHeight():
\begin{itemize}
\item transition: None
\item output: $\mathit{out := Height}$
\item exception: None
\end{itemize}

\noindent setAge(newAge):
\begin{itemize}
\item transition: $\mathit{Age := newAge}$
\item output: None
\item exception: $\neg$ isValidString(newAge) $\implies$
IllegalArgumentException
\end{itemize}

\noindent getAge():
\begin{itemize}
\item transition: None
\item output: $\mathit{out := Age}$
\item exception: None
\end{itemize}

\subsubsection{Local Functions}
ValidCharacters = \{``1", ``2", ``3", ``4", ``5", ``6", ``7"
, ``8", ``9", ``0", ``."\}\\

\noindent isValidString(S): String $\rightarrow$ $\mathbb{B}$ \\

\noindent isValidString(S) = $\forall(i : \mathbb{Z} | 0 \leq
i < |S| : S[i] \in $ ValidCharacters)
 %%%%%%%%%%%%%%%%%%%%% Beech ends %%%%%%%%%%%%%

\newpage

%%%%%%%%%%%%%%%%%%% Birch starts %%%%%%%%%%%
\renewcommand{\tn}{Birch }
\renewcommand{\tmn}{Birch}
\renewcommand{\constn}{Birch}

\section{MIS of \tn (\mref{Model7})}

\subsection{Module}
\tmn

\subsection{Uses}
None

\subsection{Syntax}
\subsubsection{Exported Constants}
None
\subsubsection{Exported Access Programs}

\begin{center}
\begin{tabular}{|l|l|l| p{5cm}|}
\hline
\textbf{Name} & \textbf{In} & \textbf{Out} & \textbf{Exceptions} \\
\hline
new \tmn & & \tmn & \\
\hline
getTreeName & &String & \\
\hline 
setDensity & String & & IllegalArgumentException\\
\hline
getDensity & & String & \\
\hline
setDBH & String & & IllegalArgumentException\\
\hline
getDBH & & String & \\
\hline
setHeight & String & & IllegalArgumentException\\
\hline
getHeight & & String & \\
\hline
setAge & String & & IllegalArgumentException\\
\hline
getAge & & String & \\
\hline
\end{tabular}
\end{center}

\subsection{Semantics}

\subsubsection{State Variables}
$\mathit{Treename: String}$\\
$\mathit{Density: String}$\\
$\mathit{DBH: String}$\\
$\mathit{Height: String}$\\
$\mathit{Age: String}$\\

\subsubsection{Environment Variables}
None

\subsubsection{Assumptions}
None

\subsubsection{Access Routine Semantics}

\noindent new \tmn():
\begin{itemize}
\item transition: $\mathit{Treename, Density, DBH, Height,
Age := ``\constn", ``", ``", ``", ``"}$
\item output: $\mathit{out := self}$
\item exception: None 
\end{itemize}
 
getTreeName():
\begin{itemize}
\item transition: None
\item output: $\mathit{out := Treename}$
\item exception: None
\end{itemize}

\noindent setDensity(newDensity):
\begin{itemize}
\item transition: $\mathit{Density := newDensity}$
\item output: None
\item exception: $\neg$ isValidString(newDensity) $\implies$
IllegalArgumentException
\end{itemize}

\noindent getDensity():
\begin{itemize}
\item transition: None
\item output: $\mathit{out := Density}$
\item exception: None
\end{itemize}

\noindent setDBH(newDBH):
\begin{itemize}
\item transition: $\mathit{DBH := newDBH}$
\item output: None
\item exception: $\neg$ isValidString(newDBH) $\implies$
IllegalArgumentException
\end{itemize}
\newpage
\noindent getDBH():
\begin{itemize}
\item transition: None
\item output: $\mathit{out := DBH}$
\item exception: None
\end{itemize}

\noindent setHeight(newHeight):
\begin{itemize}
\item transition: $\mathit{Height := newHeight}$
\item output: None
\item exception: $\neg$ isValidString(newHeight) $\implies$
IllegalArgumentException
\end{itemize}

\noindent getHeight():
\begin{itemize}
\item transition: None
\item output: $\mathit{out := Height}$
\item exception: None
\end{itemize}

\noindent setAge(newAge):
\begin{itemize}
\item transition: $\mathit{Age := newAge}$
\item output: None
\item exception: $\neg$ isValidString(newAge) $\implies$
IllegalArgumentException
\end{itemize}

\noindent getAge():
\begin{itemize}
\item transition: None
\item output: $\mathit{out := Age}$
\item exception: None
\end{itemize}

\subsubsection{Local Functions}
ValidCharacters = \{``1", ``2", ``3", ``4", ``5", ``6", ``7"
, ``8", ``9", ``0", ``."\}\\

\noindent isValidString(S): String $\rightarrow$ $\mathbb{B}$ \\

\noindent isValidString(S) = $\forall(i : \mathbb{Z} | 0 \leq
i < |S| : S[i] \in $ ValidCharacters)
 %%%%%%%%%%%%%%%%%%%%% Birch ends %%%%%%%%%%%%%

\newpage

%%%%%%%%%%%%%%%%%%% White Pine starts %%%%%%%%%%%
\renewcommand{\tn}{White Pine }
\renewcommand{\tmn}{WhitePine}
\renewcommand{\constn}{White\ Pine}

\section{MIS of \tn (\mref{Model8})}

\subsection{Module}
\tmn

\subsection{Uses}
None

\subsection{Syntax}
\subsubsection{Exported Constants}
None
\subsubsection{Exported Access Programs}

\begin{center}
\begin{tabular}{|l|l|l| p{5cm}|}
\hline
\textbf{Name} & \textbf{In} & \textbf{Out} & \textbf{Exceptions} \\
\hline
new \tmn & & \tmn & \\
\hline
getTreeName & &String & \\
\hline 
setDensity & String & & IllegalArgumentException\\
\hline
getDensity & & String & \\
\hline
setDBH & String & & IllegalArgumentException\\
\hline
getDBH & & String & \\
\hline
setHeight & String & & IllegalArgumentException\\
\hline
getHeight & & String & \\
\hline
setAge & String & & IllegalArgumentException\\
\hline
getAge & & String & \\
\hline
\end{tabular}
\end{center}

\subsection{Semantics}

\subsubsection{State Variables}
$\mathit{Treename: String}$\\
$\mathit{Density: String}$\\
$\mathit{DBH: String}$\\
$\mathit{Height: String}$\\
$\mathit{Age: String}$\\

\subsubsection{Environment Variables}
None

\subsubsection{Assumptions}
None

\subsubsection{Access Routine Semantics}

\noindent new \tmn():
\begin{itemize}
\item transition: $\mathit{Treename, Density, DBH, Height,
Age := ``\constn", ``", ``", ``", ``"}$
\item output: $\mathit{out := self}$
\item exception: None 
\end{itemize}

getTreeName():
\begin{itemize}
\item transition: None
\item output: $\mathit{out := Treename}$
\item exception: None
\end{itemize}

\noindent setDensity(newDensity):
\begin{itemize}
\item transition: $\mathit{Density := newDensity}$
\item output: None
\item exception: $\neg$ isValidString(newDensity) $\implies$
IllegalArgumentException
\end{itemize}

\noindent getDensity():
\begin{itemize}
\item transition: None
\item output: $\mathit{out := Density}$
\item exception: None
\end{itemize}

\noindent setDBH(newDBH):
\begin{itemize}
\item transition: $\mathit{DBH := newDBH}$
\item output: None
\item exception: $\neg$ isValidString(newDBH) $\implies$
IllegalArgumentException
\end{itemize}
\newpage
\noindent getDBH():
\begin{itemize}
\item transition: None
\item output: $\mathit{out := DBH}$
\item exception: None
\end{itemize}

\noindent setHeight(newHeight):
\begin{itemize}
\item transition: $\mathit{Height := newHeight}$
\item output: None
\item exception: $\neg$ isValidString(newHeight) $\implies$
IllegalArgumentException
\end{itemize}

\noindent getHeight():
\begin{itemize}
\item transition: None
\item output: $\mathit{out := Height}$
\item exception: None
\end{itemize}

\noindent setAge(newAge):
\begin{itemize}
\item transition: $\mathit{Age := newAge}$
\item output: None
\item exception: $\neg$ isValidString(newAge) $\implies$
IllegalArgumentException
\end{itemize}

\noindent getAge():
\begin{itemize}
\item transition: None
\item output: $\mathit{out := Age}$
\item exception: None
\end{itemize}

\subsubsection{Local Functions}
ValidCharacters = \{``1", ``2", ``3", ``4", ``5", ``6", ``7"
, ``8", ``9", ``0", ``."\}\\

\noindent isValidString(S): String $\rightarrow$ $\mathbb{B}$ \\

\noindent isValidString(S) = $\forall(i : \mathbb{Z} | 0 \leq
i < |S| : S[i] \in $ ValidCharacters)
 %%%%%%%%%%%%%%%%%%%%% White Pine ends %%%%%%%%%%%%%

 \newpage

%%%%%%%%%%%%%%%%%%% Red Maple starts %%%%%%%%%%%
\renewcommand{\tn}{Red Maple }
\renewcommand{\tmn}{RedMaple}
\renewcommand{\constn}{Red\ Maple}

\section{MIS of \tn (\mref{Model9})}

\subsection{Module}
\tmn

\subsection{Uses}
None

\subsection{Syntax}
\subsubsection{Exported Constants}
None
\subsubsection{Exported Access Programs}

\begin{center}
\begin{tabular}{|l|l|l| p{5cm}|}
\hline
\textbf{Name} & \textbf{In} & \textbf{Out} & \textbf{Exceptions} \\
\hline
new \tmn & & \tmn & \\
\hline
getTreeName & &String & \\
\hline 
setDensity & String & & IllegalArgumentException\\
\hline
getDensity & & String & \\
\hline
setDBH & String & & IllegalArgumentException\\
\hline
getDBH & & String & \\
\hline
setHeight & String & & IllegalArgumentException\\
\hline
getHeight & & String & \\
\hline
setAge & String & & IllegalArgumentException\\
\hline
getAge & & String & \\
\hline
\end{tabular}
\end{center}

\subsection{Semantics}

\subsubsection{State Variables}
$\mathit{Treename: String}$\\
$\mathit{Density: String}$\\
$\mathit{DBH: String}$\\
$\mathit{Height: String}$\\
$\mathit{Age: String}$\\

\subsubsection{Environment Variables}
None

\subsubsection{Assumptions}
None

\subsubsection{Access Routine Semantics}

\noindent new \tmn():
\begin{itemize}
\item transition: $\mathit{Treename, Density, DBH, Height,
Age := ``\constn", ``", ``", ``", ``"}$
\item output: $\mathit{out := self}$
\item exception: None 
\end{itemize}

getTreeName():
\begin{itemize}
\item transition: None
\item output: $\mathit{out := Treename}$
\item exception: None
\end{itemize}

\noindent setDensity(newDensity):
\begin{itemize}
\item transition: $\mathit{Density := newDensity}$
\item output: None
\item exception: $\neg$ isValidString(newDensity) $\implies$
IllegalArgumentException
\end{itemize}

\noindent getDensity():
\begin{itemize}
\item transition: None
\item output: $\mathit{out := Density}$
\item exception: None
\end{itemize}

\noindent setDBH(newDBH):
\begin{itemize}
\item transition: $\mathit{DBH := newDBH}$
\item output: None
\item exception: $\neg$ isValidString(newDBH) $\implies$
IllegalArgumentException
\end{itemize}
\newpage
\noindent getDBH():
\begin{itemize}
\item transition: None
\item output: $\mathit{out := DBH}$
\item exception: None
\end{itemize}

\noindent setHeight(newHeight):
\begin{itemize}
\item transition: $\mathit{Height := newHeight}$
\item output: None
\item exception: $\neg$ isValidString(newHeight) $\implies$
IllegalArgumentException
\end{itemize}

\noindent getHeight():
\begin{itemize}
\item transition: None
\item output: $\mathit{out := Height}$
\item exception: None
\end{itemize}

\noindent setAge(newAge):
\begin{itemize}
\item transition: $\mathit{Age := newAge}$
\item output: None
\item exception: $\neg$ isValidString(newAge) $\implies$
IllegalArgumentException
\end{itemize}

\noindent getAge():
\begin{itemize}
\item transition: None
\item output: $\mathit{out := Age}$
\item exception: None
\end{itemize}

\subsubsection{Local Functions}
ValidCharacters = \{``1", ``2", ``3", ``4", ``5", ``6", ``7"
, ``8", ``9", ``0", ``."\}\\

\noindent isValidString(S): String $\rightarrow$ $\mathbb{B}$ \\

\noindent isValidString(S) = $\forall(i : \mathbb{Z} | 0 \leq
i < |S| : S[i] \in $ ValidCharacters)
 %%%%%%%%%%%%%%%%%%%%% Red Maple ends %%%%%%%%%%%%%

\newpage

%%%%%%%%%%%%%%%%%%% Red Oak starts %%%%%%%%%%%
\renewcommand{\tn}{Red Oak }
\renewcommand{\tmn}{RedOak}
\renewcommand{\constn}{Red\ Oak}

\section{MIS of \tn (\mref{Model10})}

\subsection{Module}
\tmn

\subsection{Uses}
None

\subsection{Syntax}
\subsubsection{Exported Constants}
None
\subsubsection{Exported Access Programs}

\begin{center}
\begin{tabular}{|l|l|l| p{5cm}|}
\hline
\textbf{Name} & \textbf{In} & \textbf{Out} & \textbf{Exceptions} \\
\hline
new \tmn & & \tmn & \\
\hline
getTreeName & &String & \\
\hline 
setDensity & String & & IllegalArgumentException\\
\hline
getDensity & & String & \\
\hline
setDBH & String & & IllegalArgumentException\\
\hline
getDBH & & String & \\
\hline
setHeight & String & & IllegalArgumentException\\
\hline
getHeight & & String & \\
\hline
setAge & String & & IllegalArgumentException\\
\hline
getAge & & String & \\
\hline
\end{tabular}
\end{center}

\subsection{Semantics}

\subsubsection{State Variables}
$\mathit{Treename: String}$\\
$\mathit{Density: String}$\\
$\mathit{DBH: String}$\\
$\mathit{Height: String}$\\
$\mathit{Age: String}$\\

\subsubsection{Environment Variables}
None

\subsubsection{Assumptions}
None

\subsubsection{Access Routine Semantics}

\noindent new \tmn():
\begin{itemize}
\item transition: $\mathit{Treename, Density, DBH, Height,
Age := ``\constn", ``", ``", ``", ``"}$
\item output: $\mathit{out := self}$
\item exception: None 
\end{itemize}

getTreeName():
\begin{itemize}
\item transition: None
\item output: $\mathit{out := Treename}$
\item exception: None
\end{itemize}

\noindent setDensity(newDensity):
\begin{itemize}
\item transition: $\mathit{Density := newDensity}$
\item output: None
\item exception: $\neg$ isValidString(newDensity) $\implies$
IllegalArgumentException
\end{itemize}

\noindent getDensity():
\begin{itemize}
\item transition: None
\item output: $\mathit{out := Density}$
\item exception: None
\end{itemize}

\noindent setDBH(newDBH):
\begin{itemize}
\item transition: $\mathit{DBH := newDBH}$
\item output: None
\item exception: $\neg$ isValidString(newDBH) $\implies$
IllegalArgumentException
\end{itemize}
\newpage
\noindent getDBH():
\begin{itemize}
\item transition: None
\item output: $\mathit{out := DBH}$
\item exception: None
\end{itemize}

\noindent setHeight(newHeight):
\begin{itemize}
\item transition: $\mathit{Height := newHeight}$
\item output: None
\item exception: $\neg$ isValidString(newHeight) $\implies$
IllegalArgumentException
\end{itemize}

\noindent getHeight():
\begin{itemize}
\item transition: None
\item output: $\mathit{out := Height}$
\item exception: None
\end{itemize}

\noindent setAge(newAge):
\begin{itemize}
\item transition: $\mathit{Age := newAge}$
\item output: None
\item exception: $\neg$ isValidString(newAge) $\implies$
IllegalArgumentException
\end{itemize}

\noindent getAge():
\begin{itemize}
\item transition: None
\item output: $\mathit{out := Age}$
\item exception: None
\end{itemize}

\subsubsection{Local Functions}
ValidCharacters = \{``1", ``2", ``3", ``4", ``5", ``6", ``7"
, ``8", ``9", ``0", ``."\}\\

\noindent isValidString(S): String $\rightarrow$ $\mathbb{B}$ \\

\noindent isValidString(S) = $\forall(i : \mathbb{Z} | 0 \leq
i < |S| : S[i] \in $ ValidCharacters)
 %%%%%%%%%%%%%%%%%%%%% Red Oak ends %%%%%%%%%%%%%

\newpage

%%%%%%%%%%%%%%%%%%% Environmental Data starts %%%%%%%%%%%
\section{MIS of Environmental Data (\mref{Model11})}

\subsection{Module}
EnvData

\subsection{Uses}
None

\subsection{Syntax}
\subsubsection{Exported Constants}
None
\subsubsection{Exported Access Programs}

\begin{center}
\begin{tabular}{|l|l|l| p{5cm}|}
\hline
\textbf{Name} & \textbf{In} & \textbf{Out} & \textbf{Exceptions} \\
\hline
new EnvData & & EnvData & \\
\hline

setHumility & String & & IllegalArgumentException\\
\hline
getHumility & &String & \\
\hline

setTemp & String & & IllegalArgumentException\\
\hline
getTemp & &String & \\
\hline

setSC & String & & IllegalArgumentException\\
\hline
getSC & &String & \\
\hline

setSN & String & & IllegalArgumentException\\
\hline
getSN & &String & \\
\hline

setLAI & String & & IllegalArgumentException\\
\hline
getLAI & &String & \\
\hline
\end{tabular}
\end{center}

\subsection{Semantics}

\subsubsection{State Variables}
$\mathit{Humility: String}$\\
$\mathit{Temp: String}$\\
$\mathit{SC: String}$\\
$\mathit{SN: String}$\\
$\mathit{LAI: String}$\\

\subsubsection{Environment Variables}
None

\subsubsection{Assumptions}
None

\subsubsection{Access Routine Semantics}

\noindent new EnvData():
\begin{itemize}
\item transition: $\mathit{Humility, Temp, SC, SN, LAI := ``", ``", ``", ``", ``"}$
\item output: $\mathit{out := self}$
\item exception: None 
\end{itemize}


\newcommand{\attr}{Humility}
\noindent get\attr():
\begin{itemize}
\item transition: None
\item output: $\mathit{out := \attr}$
\item exception: None
\end{itemize}

set\attr(new\attr):
\begin{itemize}
\item transition: $\mathit{\attr := new\attr}$
\item output: None
\item exception: $\neg$ isValidString(new\attr) $\implies$
IllegalArgumentException
\end{itemize}


\renewcommand{\attr}{Temp}
\noindent get\attr():
\begin{itemize}
\item transition: None
\item output: $\mathit{out := \attr}$
\item exception: None
\end{itemize}

\noindent set\attr(new\attr):
\begin{itemize}
\item transition: $\mathit{\attr := new\attr}$
\item output: None
\item exception: $\neg$ isValidString(new\attr) $\implies$
IllegalArgumentException
\end{itemize}


\renewcommand{\attr}{SC}
\noindent get\attr():
\begin{itemize}
\item transition: None
\item output: $\mathit{out := \attr}$
\item exception: None
\end{itemize}

\noindent set\attr(new\attr):
\begin{itemize}
\item transition: $\mathit{\attr := new\attr}$
\item output: None
\item exception: $\neg$ isValidString(new\attr) $\implies$
IllegalArgumentException
\end{itemize}


\renewcommand{\attr}{SN}
\noindent get\attr():
\begin{itemize}
\item transition: None
\item output: $\mathit{out := \attr}$
\item exception: None
\end{itemize}

\noindent set\attr(new\attr):
\begin{itemize}
\item transition: $\mathit{\attr := new\attr}$
\item output: None
\item exception: $\neg$ isValidString(new\attr) $\implies$
IllegalArgumentException
\end{itemize}


\renewcommand{\attr}{LAI}
\noindent get\attr():
\begin{itemize}
\item transition: None
\item output: $\mathit{out := \attr}$
\item exception: None
\end{itemize}

\noindent set\attr(new\attr):
\begin{itemize}
\item transition: $\mathit{\attr := new\attr}$
\item output: None
\item exception: $\neg$ isValidString(new\attr) $\implies$
IllegalArgumentException
\end{itemize}


\subsubsection{Local Functions}
ValidCharacters = \{``1", ``2", ``3", ``4", ``5", ``6", ``7"
, ``8", ``9", ``0", ``."\}\\

\noindent isValidString(S): String $\rightarrow$ $\mathbb{B}$ \\

\noindent isValidString(S) = $\forall(i : \mathbb{Z} | 0 \leq
i < |S| : S[i] \in $ ValidCharacters)
 %%%%%%%%%%%%%%%%%%%%% Environmental data ends %%%%%%%%%%%%%

\newpage

%%%%%%%%%%%%%%%% Plot Data starts %%%%%%%%%%%%%%%
\section{MIS of Plot Data (\mref{Model12})}

\subsection{Module}
PlotData

\subsection{Uses}
\mref{Model4}, \mref{Model5}, \mref{Model6}, \mref{Model7},
\mref{Model8}, \mref{Model9}, \mref{Model10}, \mref{Model11}    

\subsection{Syntax}
\subsubsection{Exported Constants}
None
\subsubsection{Exported Access Programs}

\begin{center}
\begin{tabular}{|l|l|l| p{5cm}|}
\hline
\textbf{Name} & \textbf{In} & \textbf{Out} & \textbf{Exceptions} \\
\hline
new PlotData & & PlotData & \\
\hline

setRedPineObj & RedPine  & & \\
\hline
getRedPineObj & & RedPine & \\
\hline

setOakObj & Oak  & & \\
\hline
getOakObj & & Oak & \\
\hline

setBeechObj & Beech  & & \\
\hline
getBeechObj & & Beech & \\
\hline

setBirchObj & Birch  & & \\
\hline
getBirchObj & & Birch & \\
\hline

setWhitePineObj & WhitePine  & & \\
\hline
getWhitePineObj & & WhitePine & \\
\hline

setRedMapleObj & RedMaple  & & \\
\hline
getRedMapleObj & & RedMaple & \\
\hline

setRedOakObj & RedOak  & & \\
\hline
getRedOakObj & & RedOak & \\
\hline

setEnvDataObj & EnvData  & & \\
\hline
getEnvDataObj & & EnvData & \\
\hline

\end{tabular}
\end{center}

\newpage

\subsection{Semantics}

\subsubsection{State Variables}
$\mathit{RedPineObj: RedPine}$\\
$\mathit{OakObj: Oak}$\\
$\mathit{BeechObj: Beech}$\\
\renewcommand{\attr}{Birch}
$\mathit{\attr Obj: \attr}$\\
\renewcommand{\attr}{WhitePine}
$\mathit{\attr Obj: \attr}$\\
\renewcommand{\attr}{RedMaple}
$\mathit{\attr Obj: \attr}$\\
\renewcommand{\attr}{RedOak}
$\mathit{\attr Obj: \attr}$\\
\renewcommand{\attr}{EnvData}
$\mathit{\attr Obj: \attr}$\\

\subsubsection{Environment Variables}
None

\subsubsection{Assumptions}
None

\subsubsection{Access Routine Semantics}

\noindent new PlotData():
\begin{itemize}
\item transition: 
\begin{itemize}
\item $\mathit{RedPineObj, OakObj, BeechObj, 
BirchObj := null, null, null, null}$

\item $\mathit{WhitePineObj, RedMapleObj, RedOakObj, EvnDataObj
:= null, null, null, null}$
\end{itemize}

\item output: $\mathit{out := self}$
\item exception: None 
\end{itemize}


\renewcommand{\attr}{RedPineObj}
\noindent get\attr():
\begin{itemize}
\item transition: None
\item output: $\mathit{out := \attr}$
\item exception: None
\end{itemize}

\noindent set\attr(new\attr):
\begin{itemize}
\item transition: $\mathit{\attr := new\attr}$
\item output: None
\item exception: None
\end{itemize}


\renewcommand{\attr}{OakObj}
\noindent get\attr():
\begin{itemize}
\item transition: None
\item output: $\mathit{out := \attr}$
\item exception: None
\end{itemize}

\noindent set\attr(new\attr):
\begin{itemize}
\item transition: $\mathit{\attr := new\attr}$
\item output: None
\item exception: None
\end{itemize}


\renewcommand{\attr}{BeechObj}
\noindent get\attr():
\begin{itemize}
\item transition: None
\item output: $\mathit{out := \attr}$
\item exception: None
\end{itemize}

\noindent set\attr(new\attr):
\begin{itemize}
\item transition: $\mathit{\attr := new\attr}$
\item output: None
\item exception: None
\end{itemize}

\renewcommand{\attr}{BirchObj}
\noindent get\attr():
\begin{itemize}
\item transition: None
\item output: $\mathit{out := \attr}$
\item exception: None
\end{itemize}

\noindent set\attr(new\attr):
\begin{itemize}
\item transition: $\mathit{\attr := new\attr}$
\item output: None
\item exception: None
\end{itemize}


\renewcommand{\attr}{WhitePineObj}
\noindent get\attr():
\begin{itemize}
\item transition: None
\item output: $\mathit{out := \attr}$
\item exception: None
\end{itemize}

\noindent set\attr(new\attr):
\begin{itemize}
\item transition: $\mathit{\attr := new\attr}$
\item output: None
\item exception: None
\end{itemize}


\renewcommand{\attr}{RedMapleObj}
\noindent get\attr():
\begin{itemize}
\item transition: None
\item output: $\mathit{out := \attr}$
\item exception: None
\end{itemize}

\noindent set\attr(new\attr):
\begin{itemize}
\item transition: $\mathit{\attr := new\attr}$
\item output: None
\item exception: None
\end{itemize}


\renewcommand{\attr}{RedOakObj}
\noindent get\attr():
\begin{itemize}
\item transition: None
\item output: $\mathit{out := \attr}$
\item exception: None
\end{itemize}

\noindent set\attr(new\attr):
\begin{itemize}
\item transition: $\mathit{\attr := new\attr}$
\item output: None
\item exception: None
\end{itemize}


\renewcommand{\attr}{EnvDataObj}
\noindent get\attr():
\begin{itemize}
\item transition: None
\item output: $\mathit{out := \attr}$
\item exception: None
\end{itemize}

\noindent set\attr(new\attr):
\begin{itemize}
\item transition: $\mathit{\attr := new\attr}$
\item output: None
\item exception: None
\end{itemize}


\subsubsection{Local Functions}
None
%%%%%%%%%%%%%%%% Plot Data Ends %%%%%%%%%%%%%%%%%

\newpage

%%%%%%%%%%%%%%%%%  First Person Player starts %%%%%%%
\section{MIS of First Person Player (\mref{Model13})}

\subsection{Module}
FirstPersonPlayer

\subsection{Uses}
Character Controller Module from Unity

\subsection{Syntax}
This is a module provided by UnityEngine.UI. Please click
\href{https://docs.unity3d.com/ScriptReference/CharacterController.html}{here} to check offical document from Unity. We have 
designed a controller for this module. The controller is 
PlayerMovement(\mref{Controller4}).
\subsection{Semantics}
This is a module provided by UnityEngine.UI. Please click
\href{https://docs.unity3d.com/ScriptReference/CharacterController.html}{here} to check offical document from Unity. We have 
designed a controller for this module. The controller is 
PlayerMovement(\mref{Controller4}).

%%%%%%%%%%%%%%%%%% First Person Player ends %%%%%%%%%

\newpage

%%%%%%%%%%%%%%%%%%%%%%%%%%%% Json File Start %%%%%%%%%
\section{MIS of Json File (\mref{Model14})}

\subsection{Module}
JsonFile. This is not a typical class. This section only aims
to show how JSON files are organized formally.

\subsection{Local Type}
$X = tuple(key:String,\ value:String)$ $\land$ 
isValidString(value) \\
$S : set\ of\ X$\\
$TreeANDEnvData = tuple(key:String,\ values: S)$

\subsection{State Variables}
$JsonFile : set\ of\ TreeANDEnvData$

\newcounter{var}

\subsection{Example}
\begin{itemize}
    \item 
First, define all the tuples that have type $X$.
\begin{itemize}
    \item $x_1 = ("DBH", "10")\ :X$
    \item $x_2 = ("Age", "10")\ :X$
    \item $x_3 = ("Height", "10")\ :X$
    \item $x_4 = ("Density", "10")\ :X$
    \vspace{0.5cm}

    \item $x_5 = ("DBH", "20")\ :X$
    \item $x_6 = ("Age", "20")\ :X$
    \item $x_7 = ("Height", "20")\ :X$
    \item $x_8 = ("Density", "20")\ :X$
    \vspace{0.5cm}
    
    \item $x_9 = ("DBH", "30")\ :X$
    \item $x_{10} = ("Age", "30")\ :X$
    \item $x_{11} = ("Height", "30")\ :X$
    \item $x_{12} = ("Density", "30")\ :X$
    \vspace{0.5cm}
    
    \renewcommand{\attr}{40}
    \item $x_{13} = ("DBH", "\attr")\ :X$
    \item $x_{14} = ("Age", "\attr")\ :X$
    \item $x_{15} = ("Height", "\attr")\ :X$
    \item $x_{16} = ("Density", "\attr")\ :X$
    \vspace{0.5cm}

    \renewcommand{\attr}{50}
    \item $x_{17} = ("DBH", "\attr")\ :X$
    \item $x_{18} = ("Age", "\attr")\ :X$
    \item $x_{19} = ("Height", "\attr")\ :X$
    \item $x_{20} = ("Density", "\attr")\ :X$
    \vspace{0.5cm}

    \renewcommand{\attr}{60}
    \item $x_{21} = ("DBH", "\attr")\ :X$
    \item $x_{22} = ("Age", "\attr")\ :X$
    \item $x_{23} = ("Height", "\attr")\ :X$
    \item $x_{24} = ("Density", "\attr")\ :X$
    \vspace{0.5cm}

    \renewcommand{\attr}{70}
    \item $x_{25} = ("DBH", "\attr")\ :X$
    \item $x_{26} = ("Age", "\attr")\ :X$
    \item $x_{27} = ("Height", "\attr")\ :X$
    \item $x_{28} = ("Density", "\attr")\ :X$
    \vspace{0.5cm}

    
    \item $x_{29} = ("Humility", "10")\ :X$
    \item $x_{30} = ("Temperature", "20")\ :X$
    \item $x_{31} = ("SC", "10")\ :X$
    \item $x_{32} = ("SN", "95")\ :X$
    \item $x_{33} = ("LAI", "95")\ :X$
\end{itemize}

    \item
Second, define all the sets that have type $S$
\begin{itemize}
    \item $s_1 = \{x_1, x_2, x_3, x_4\}\ :S$
    \item $s_2 = \{x_5, x_6, x_7, x_8\}\ :S$
    \item $s_3 = \{x_9, x_{10}, x_{11}, x_{12}\}\ :S$
    \item $s_4 = \{x_{13}, x_{14}, x_{15}, x_{16}\}\ :S$
    \item $s_5 = \{x_{17}, x_{18}, x_{19}, x_{20}\}\ :S$
    \item $s_6 = \{x_{21}, x_{22}, x_{23}, x_{24}\}\ :S$
    \item $s_7 = \{x_{25}, x_{26}, x_{27}, x_{28}\}\ :S$
    \item $s_8 = \{x_{29}, x_{30}, x_{31}, x_{32}, x_{33}\}\ :S$
\end{itemize}
    \newpage
    \item
Third, define all the tuples that have type $TreeANDEnvData$.
\begin{itemize}
    \item $d_1 = ("RedPineData", s_1):\ TreeANDEnvData$
    \item $d_2 = ("OakData", s_2):\ TreeANDEnvData$
    \item $d_3 = ("BeechData", s_3):\ TreeANDEnvData$
    \item $d_4 = ("BirchData", s_4):\ TreeANDEnvData$
    \item $d_5 = ("WhitePineData", s_5):\ TreeANDEnvData$
    \item $d_6 = ("RedMapleData", s_6):\ TreeANDEnvData$
    \item $d_7 = ("RedOakData", s_7):\ TreeANDEnvData$
    \item $d_8 = ("EnvData", s_8):\ TreeANDEnvData$
\end{itemize}
    \item
    Finally, $JsonFile = \{d_1, d_2, d_3, d_4, d_5, d_6,
    d_7, d_8\}.$
\end{itemize}

\subsection{Local Functions}
ValidCharacters = \{``1", ``2", ``3", ``4", ``5", ``6", ``7"
, ``8", ``9", ``0", ``."\}\\

\noindent isValidString(S): String $\rightarrow$ $\mathbb{B}$ \\

\noindent isValidString(S) = $\forall(i : \mathbb{Z} | 0 \leq
i < |S| : S[i] \in $ ValidCharacters)
%%%%%%%%%%%%%%%%%%%%%%%%%%%% Json File Ends %%%%%%%%%%

\newpage

%%%%%%%%%%%%%%%%% MainPage starts %%%%%%%%%%%%%%
\section{MIS of Main Page (\mref{Viwer1})}

\subsection{Module}
MainPageDisplay

\subsection{Uses}
\mref{Controller6} , 
UnityEngine.UI

\subsection{Syntax}
\subsubsection{Exported Constants}
None
\subsubsection{Exported Access Programs}
None

\subsection{Semantics}
This module is used to display the UI of the homepage.  You can refer to Unity Canvas Documentation by clicking \href{https://docs.unity3d.com/Packages/com.unity.ugui@1.0/manual/class-Canvas.html}{here}.
\subsubsection{State Variables}
None
\subsubsection{Environment Variables}
None
\subsubsection{Assumptions}
None
\subsubsection{Access Routine Semantics}
None
\subsubsection{Local Functions}
None
%%%%%%%%%%%%%%%% Mainpage ends %%%%%%%%%%%%%


\newpage


\newcommand{\bref}{\href{https://docs.unity3d.com/Packages/com.unity.ugui@1.0/manual/script-Button.html}{here}}
%%%%%%%%%%%%%%%%% StartButton starts %%%%%%%%%%%%%%
\section{MIS of Start Button (\mref{Viwer2})}

\subsection{Module}
StartButton

\subsection{Uses}
\mref{Controller6} , 
UnityEngine.UI

\subsection{Syntax}
\subsubsection{Exported Constants}
None
\subsubsection{Exported Access Programs}
None

\subsection{Semantics}
This module is used to display the UI of the StartButton. You can refer to Unity Button Documentation by clicking \bref.
\subsubsection{State Variables}
None
\subsubsection{Environment Variables}
None
\subsubsection{Assumptions}
None
\subsubsection{Access Routine Semantics}
None
\subsubsection{Local Functions}
None
%%%%%%%%%%%%%%%% StartButton ends %%%%%%%%%%%%%


\newpage

%%%%%%%%%%%%%% Instruction button starts %%%%%%%%%%%%%
\section{MIS of Instruction Button (\mref{Viwer3})}

\subsection{Module}
InstructionButton

\subsection{Uses}
\mref{Controller7}  , UnityEngine.UI

\subsection{Syntax}
\subsubsection{Exported Constants}
None
\subsubsection{Exported Access Programs}
None

\subsection{Semantics}
This module is used to display the UI of the InstructionButton.You can refer to Unity Button Documentation by clicking \bref.
\subsubsection{State Variables}
None
\subsubsection{Environment Variables}
None
\subsubsection{Assumptions}
None
\subsubsection{Access Routine Semantics}
None
\subsubsection{Local Functions}
None
%%%%%%%%%%%% InstructionButton ends %%%%%%%%%%%%%

\newpage

%%%%%%%%%%%%%%%% Contact Us Button %%%%%%%%%%%%%%%%
\section{MIS of Contact Us Button (\mref{Viwer4})}

\subsection{Module}
ContactUsButton

\subsection{Uses}
\mref{Controller8}  , UnityEngine.UI

\subsection{Syntax}
\subsubsection{Exported Constants}
None
\subsubsection{Exported Access Programs}
None

\subsection{Semantics}
This module is used to display the UI of the ContactUsButton.You can refer to Unity Button Documentation by clicking \bref.
\subsubsection{State Variables}
None
\subsubsection{Environment Variables}
None
\subsubsection{Assumptions}
None
\subsubsection{Access Routine Semantics}
None
\subsubsection{Local Functions}
None
%%%%%%%%%%%%% contact us button ends %%%%%%%%%%%%%%%%%

\newpage

%%%%%%%%%%%%%%%%%%% Quit button starts %%%%%%%%%%%%%%%
\section{MIS of Quit Button (\mref{Viwer5})}

\subsection{Module}
QuitButton

\subsection{Uses}
\mref{Controller9} ,UnityEngine.UI 

\subsection{Syntax}
\subsubsection{Exported Constants}
None
\subsubsection{Exported Access Programs}
None

\subsection{Semantics}
This module is used to display the UI of the QuitButton.You can refer to Unity Button Documentation by clicking \bref.
\subsubsection{State Variables}
None
\subsubsection{Environment Variables}
None
\subsubsection{Assumptions}
None
\subsubsection{Access Routine Semantics}
None
\subsubsection{Local Functions}
None
%%%%%%%%%%%%%%%% Quit button ends %%%%%%%%%%%%%%%%%%%

\newpage

%%%%%%%%%%%%%%%%% InstructionInfoDisplay starts %%%%%%%%%%%%%%
\section{MIS of Instruction Page (\mref{Viwer6})}

\subsection{Module}
InstructionInfoDisplay

\subsection{Uses}
\mref{Controller6} ,
UnityEngine.UI

\subsection{Syntax}
\subsubsection{Exported Constants}
None
\subsubsection{Exported Access Programs}
None

\subsection{Semantics}
This module is used to display the UI of the instruction page. 
You can refer to Unity Canvas Documentation by clicking 
\href{https://docs.unity3d.com/Packages/com.unity.ugui@1.0/manual/class-Canvas.html}{here}.
\subsubsection{State Variables}
None
\subsubsection{Environment Variables}
None
\subsubsection{Assumptions}
None
\subsubsection{Access Routine Semantics}
None
\subsubsection{Local Functions}
None
%%%%%%%%%%%%%%%% InstructionInfoDisplay ends %%%%%%%%%%%%%

\newpage

%%%%%%%%%%%%%%%%% ContactUsInfoDisplay starts %%%%%%%%%%%%%%
\section{MIS of Contact Us Page (\mref{Viwer7})}

\subsection{Module}
ContactUsInfoDisplay

\subsection{Uses}
\mref{Controller6} 
UnityEngine.UI

\subsection{Syntax}
\subsubsection{Exported Constants}
None
\subsubsection{Exported Access Programs}
None

\subsection{Semantics}
This module is used to display the UI of the Contact Us page.
You can refer to Unity Canvas Documentation by clicking 
\href{https://docs.unity3d.com/Packages/com.unity.ugui@1.0/manual/class-Canvas.html}{here}.

\subsubsection{State Variables}
None
\subsubsection{Environment Variables}
None
\subsubsection{Assumptions}
None
\subsubsection{Access Routine Semantics}
None
\subsubsection{Local Functions}
None
%%%%%%%%%%%%%%%% ContactUsInfoDisplay ends %%%%%%%%%%%%%


\newpage

%%%%%%%%%%%%%%%%% Back button starts %%%%%%%%%%%
\section{MIS of Back Button (\mref{Viwer8})}

\subsection{Module}
BackButton

\subsection{Uses}
\mref{Controller11}  , UnityEngine.UI

\subsection{Syntax}
\subsubsection{Exported Constants}
None
\subsubsection{Exported Access Programs}
None

\subsection{Semantics}
This module is used to display the UI of the BackButton.You can refer to Unity Button Documentation by clicking \bref.
\subsubsection{State Variables}
None
\subsubsection{Environment Variables}
None
\subsubsection{Assumptions}
None
\subsubsection{Access Routine Semantics}
None
\subsubsection{Local Functions}
None
%%%%%%%%%%%%%% Back button ends %%%%%%%%%%%%%%%

\newpage

%%%%%%%%%%%%%%%%% UpdateDataDisplay starts %%%%%%%%%%%%%%
\section{MIS of Update Data Page (\mref{Viwer9})}

\subsection{Module}
UpdateDataDisplay

\subsection{Uses}
\mref{Controller6} 
UnityEngine.UI

\subsection{Syntax}
\subsubsection{Exported Constants}
None
\subsubsection{Exported Access Programs}
None

\subsection{Semantics}
This module is used to display the UI of the Update Data page.
You can refer to Unity Canvas Documentation by clicking 
\href{https://docs.unity3d.com/Packages/com.unity.ugui@1.0/manual/class-Canvas.html}{here}.
\subsubsection{State Variables}
None
\subsubsection{Environment Variables}
None
\subsubsection{Assumptions}
None
\subsubsection{Access Routine Semantics}
None
\subsubsection{Local Functions}
None
%%%%%%%%%%%%%%%% UpdateDataDisplay ends %%%%%%%%%%%%%


\newpage

%%%%%%%%%%%%%%%%%%%%%%%%%%%%%%%%%%%%%%%%%%%%%%%%
\section{MIS of Environmental Data Selection Button
 (\mref{Viwer10})}

\subsection{Module}
EnvDataSelectionButton

\subsection{Uses}
\mref{Controller16}  , UnityEngine.UI

\subsection{Syntax}
\subsubsection{Exported Constants}
None
\subsubsection{Exported Access Programs}
None

\subsection{Semantics}
This module is used to display the UI of the EnvDataSelectionButton.You can refer to Unity Button Documentation by clicking \bref.
\subsubsection{State Variables}
None
\subsubsection{Environment Variables}
None
\subsubsection{Assumptions}
None
\subsubsection{Access Routine Semantics}
None
\subsubsection{Local Functions}
None
%%%%%%%%%%%%%%%%%%%%%%%%%%%%%%%%%%%%%%%%%%%%%%%

\newpage

%%%%%%%%%%%%%%%%%%%%%%%%%%%%%%%%%%%%%%%%%%%%%%%
\section{MIS of Data Type Selection Button (\mref{Viwer11})}

\subsection{Module}
DataTypeSelectionButton

\subsection{Uses}
\mref{Controller17}  , UnityEngine.UI

\subsection{Syntax}
\subsubsection{Exported Constants}
None
\subsubsection{Exported Access Programs}
None

\subsection{Semantics}
This module is used to display the UI of the DataTypeSelectionButton.You can refer to Unity Button Documentation by clicking \bref.
\subsubsection{State Variables}
None
\subsubsection{Environment Variables}
None
\subsubsection{Assumptions}
None
\subsubsection{Access Routine Semantics}
None
\subsubsection{Local Functions}
None
%%%%%%%%%%%%%%%%%%%%%%%%%%%%%%%%%%%%%%%%%%%%%%%%%%

\newpage

%%%%%%%%%%%%%%%%% NewDataInputBox starts %%%%%%%%%%%%%%
\section{MIS of New Data Input Box (\mref{Viwer12})}

\subsection{Module}
NewDataInputBox

\subsection{Uses}
\mref{Controller6} ,
UnityEngine.UI

\subsection{Syntax}
\subsubsection{Exported Constants}
None
\subsubsection{Exported Access Programs}
None

\subsection{Semantics}
This module is used to display the UI of the new data input box.
You can refer to Unity Input Field Documentation by clicking 
\href{https://docs.unity3d.com/Packages/com.unity.ugui@1.0/manual/script-InputField.html}{here}
\subsubsection{State Variables}
None
\subsubsection{Environment Variables}
None
\subsubsection{Assumptions}
None
\subsubsection{Access Routine Semantics}
None
\subsubsection{Local Functions}
None
%%%%%%%%%%%%%%%% NewDataInputBox ends %%%%%%%%%%%%%


\newpage

%%%%%%%%%%%%%%%%%%%%%%%%%%%%%%%%%%%%%%%%%%%%%%%%%%
\section{MIS of Save Button (\mref{Viwer13})}

\subsection{Module}
SaveButton

\subsection{Uses}
\mref{Controller18}  , UnityEngine.UI

\subsection{Syntax}
\subsubsection{Exported Constants}
None
\subsubsection{Exported Access Programs}
None

\subsection{Semantics}
This module is used to display the UI of the SaveButton.You can refer to Unity Button Documentation by clicking \bref.
\subsubsection{State Variables}
None
\subsubsection{Environment Variables}
None
\subsubsection{Assumptions}
None
\subsubsection{Access Routine Semantics}
None
\subsubsection{Local Functions}
None
%%%%%%%%%%%%%%%%%%%%%%%%%%%%%%%%%%%%%%%%%%%%%%%%%%%

\newpage


\newcommand{\tref}{\href{https://docs.unity3d.com/Packages/com.unity.ugui@1.0/manual/script-Text.html}{here}}

%%%%%%%%%%% CurrentDataDisplay starts %%%%%%%%%%%%%%%%%%
\section{MIS of Current Data Dispaly (\mref{Viwer14})} 

\subsection{Module}
CurrentDataDisplay

\subsection{Uses}
UnityEngine.UI 

\subsection{Syntax}
\subsubsection{Exported Constants}
None
\subsubsection{Exported Access Programs}
None

\subsection{Semantics}
This module is used to display the UI of the current data. 
You can refer to Unity Text Documentation by clicking \tref.
\subsubsection{State Variables}
None
\subsubsection{Environment Variables}
None
\subsubsection{Assumptions}
None
\subsubsection{Access Routine Semantics}
None
\subsubsection{Local Functions}
None
%%%%%%%%%%%%%%%%%%%%%%%%% CurrentDataDisplay ends %%%%%%%%%%%%%

\newpage

\newcommand{\dref}{\href{https://docs.unity3d.com/Packages/com.unity.ugui@1.0/manual/script-Dropdown.html}{here}}
%%%%%%%%%%% PlotSelectionDropDown starts %%%%%%%%%%%%%%%%%%
\section{MIS of Plot Selection Drop Down (\mref{Viwer15})} 

\subsection{Module}

PlotSelection

\subsection{Uses}

\mref{Controller12}, UnityEngine.UI

\subsection{Syntax}

\subsubsection{Exported Constants}
None
\subsubsection{Exported Access Programs}
None

\subsection{Semantics}
This module is used to display the dropdown box of plot 
selection. You can refer to Unity Drop Down Documentation 
by clicking \dref.
\subsubsection{State Variables}
None
\subsubsection{Environment Variables}
None
\subsubsection{Assumptions}
None
\subsubsection{Access Routine Semantics}
None
\subsubsection{Local Functions}
None
%%%%%%%%%%%%%%%%%%%%%%%%% PlotSelectionDropDown ends %%%%%%%%%

\newpage

%%%%%%%%%%% TreeTypeSelectionDropDown starts %%%%%%%%%%%%%%%%%%
\section{MIS of Tree Type Selection Drop Down (\mref{Viwer16})} 

\subsection{Module}

TreeTypeSelection

\subsection{Uses}

\mref{Controller13} , UnityEngine.UI

\subsection{Syntax}

\subsubsection{Exported Constants}
None
\subsubsection{Exported Access Programs}
None

\subsection{Semantics}
This module is used to display the dropdown box of the tree type
selection. You can refer to Unity Drop Down Documentation 
by clicking \dref.

\subsubsection{State Variables}
None
\subsubsection{Environment Variables}
None
\subsubsection{Assumptions}
None
\subsubsection{Access Routine Semantics}
None
\subsubsection{Local Functions}
None

%%%%%%%%%%%%%%%%%%%%%%%%% TreeTypeSelectionDropDown ends %%%%%%

\newpage

%%%%%%%%%%%%%%%%%%%%%%%%%%%%%%%%%%%%%%%%%%%
\section{MIS of Update Data Button (\mref{Viwer17})}

\subsection{Module}
UpdateDataButton

\subsection{Uses}
\mref{Controller8} , UnityEngine.UI 

\subsection{Syntax}
\subsubsection{Exported Constants}
None
\subsubsection{Exported Access Programs}
None

\subsection{Semantics}
The module is used to display the UI of UpdateDataButton. You can refer to Unity Button Documentation by clicking \bref.
\subsubsection{State Variables}
None
\subsubsection{Environment Variables}
None
\subsubsection{Assumptions}
None
\subsubsection{Access Routine Semantics}
None
\subsubsection{Local Functions}
None
%%%%%%%%%%%%%%%%%%%%%%%%%%%%%%%%%%%%%%%%%%%%%%%%%%%%

\newpage

%%%%%%%%%%% ForestDisplay starts %%%%%%%%%%%%%%%%%%
\section{MIS of Forest Dispaly (\mref{Viwer18})} 

\subsection{Module}
ForestDisplay

\subsection{Uses}
UnityEngine.UI, \mref{Model1}, \mref{Model2}, \mref{Model3}

\subsection{Syntax}

\subsubsection{Exported Constants}
None
\subsubsection{Exported Access Programs}
None

\subsection{Semantics}
\subsubsection{State Variables}
This module is used to display the forest models.
\subsubsection{Environment Variables}
None
\subsubsection{Assumptions}
None
\subsubsection{Access Routine Semantics}
None
\subsubsection{Local Functions}
None
%%%%%%%%%%%%%%%%%%%%%%%%% ForestDisplay ends %%%%%%%%%%%%%

\newpage

%%%%%%%%%%%%%%%%%%%%%%%%%%%%%%%%%%%%%%%%%%%%%%%%%%%%%
\section{MIS of Show Environmental Data Button (\mref{Viwer19})}

\subsection{Module}
ShowEnvDataButton

\subsection{Uses}
\mref{Controller14}  , UnityEngine.UI

\subsection{Syntax}
\subsubsection{Exported Constants}
None
\subsubsection{Exported Access Programs}
None


\subsection{Semantics}
This module is used to display the UI of the 
ShowEnvDataButton.You can refer to Unity Button Documentation by
clicking \bref.
\subsubsection{State Variables}
None
\subsubsection{Environment Variables}
None
\subsubsection{Assumptions}
None
\subsubsection{Access Routine Semantics}
None
\subsubsection{Local Functions}
None

%%%%%%%%%%%%%%%%%%%%%%%%%%%%%%%%%%%%%%%%%%%%%%%%

\newpage

%%%%%%%%%%%%%%%%%%%%%%%%%%%%%%%%%%%%%%%%%%%%%%%%%
\section{MIS of Show Tree Parameters Button (\mref{Viwer20})}

\subsection{Module}
ShowTreeParamButton

\subsection{Uses}
\mref{Controller15}  , UnityEngine.UI

\subsection{Syntax}
\subsubsection{Exported Constants}
None
\subsubsection{Exported Access Programs}
None


\subsection{Semantics}
This module is used to display the UI of the 
ShowTreeParamButton.You can refer to Unity Button Documentation by
clicking \bref.

\subsubsection{State Variables}
None

\subsubsection{Environment Variables}
None

\subsubsection{Assumptions}
None

\subsubsection{Access Routine Semantics}
None



\subsubsection{Local Functions}
None

\newpage


%%%%%%%%%%% EnvDataDisplay starts %%%%%%%%%%%%%%%%%%
\section{MIS of Environment Data Display (\mref{Viwer21})} 

\subsection{Module}
EnvDataDisplay

\subsection{Uses}
UnityEngine.UI 

\subsection{Syntax}

\subsubsection{Exported Constants}
None
\subsubsection{Exported Access Programs}
None

\subsection{Semantics}
This module is used to display the UI of the environment data. 
You can check Unity Text Documentation by clicking \tref.
\subsubsection{State Variables}
None
\subsubsection{Environment Variables}
None

\subsubsection{Assumptions}

None

\subsubsection{Access Routine Semantics}
None

\subsubsection{Local Functions}
None
%%%%%%%%%%%%%%%%%%%%%%%%% EnvDataDisplay ends %%%%%%%%%%%%%

\newpage

%%%%%%%%%%% TreeParamDisplay starts %%%%%%%%%%%%%%%%%%
\section{MIS of Tree Parameters Display (\mref{Viwer22})} 

\subsection{Module}
TreeParamDisplay

\subsection{Uses}
UnityEngine.UI 

\subsection{Syntax}

\subsubsection{Exported Constants}
None
\subsubsection{Exported Access Programs}
None

\subsection{Semantics}
This module is used to display the UI of the tree parameters.
You can check Unity Text Documentation by clicking \tref.
\subsubsection{State Variables}
None
\subsubsection{Environment Variables}
None
\subsubsection{Assumptions}
None
\subsubsection{Access Routine Semantics}
None
\subsubsection{Local Functions}
None
%%%%%%%%%%%%%%%%%%%%%%%%% TreeparamDisplay ends %%%%%%%%%%%%%

\newpage


%%%%%%%%%%% PauseIndicatorDisplay starts %%%%%%%%%%%%%%%%%%
\section{MIS of Pause Indicator (\mref{Viwer23})} 

\subsection{Module}
PauseIndicatorDisplay

\subsection{Uses}
UnityEngine.UI 

\subsection{Syntax}

\subsubsection{Exported Constants}
None
\subsubsection{Exported Access Programs}
None

\subsection{Semantics}
This module is used to display the status of pausing. You can
check Unity Text Documentation by clicking \tref.

\subsubsection{State Variables}
None

\subsubsection{Environment Variables}

None

\subsubsection{Assumptions}

None

\subsubsection{Access Routine Semantics}
None

\subsubsection{Local Functions}
None
%%%%%%%%%%%%%%%%% PauseIndicatorDisplay ends %%%%%%%%%%

\newpage

%%%%%%%%%%%% reader starts
\section{MIS of JSON File Reader Module  (\mref{Controller1})}

\subsection{Module}

JsonFileReader

\subsection{Uses}

System.Collections\\
System.Collections.Generic\\
UnityEngine\\
System.IO\\
UnityEngine.UI\\
\mref{Viwer21}\\
\mref{Viwer22}\\
\mref{Model12}\\
\mref{Model4}\\
\mref{Model5}\\
\mref{Model6}\\
\mref{Model7}\\
\mref{Model8}\\
\mref{Model9}\\
\mref{Model10}\\
\mref{Model11}\\


\subsection{Syntax}
\subsubsection{Exported Constants}
None
\subsubsection{Exported Access Programs}
\begin{center}
\begin{tabular}{|l|l|l|p{5cm}|}
\hline
\textbf{Name} & \textbf{In} & \textbf{Out} & \textbf{Exceptions} \\
\hline
Awake &  &  &  \\
\hline
Start& & & \\ 
\hline
readFile& $\mathbb{Z}$ & & \\
\hline
\end{tabular}
\end{center}

\subsection{Semantics}
\subsubsection{State Variables}
treeParamDisplay: TreeParamDisplay\\
envDataDisplay: EnvDataDisplay\\
dataModelObj: DataModel\\
jsonModelObj: JsonModel\\
plotNumber: $\mathbb{Z}$\\
filePath: string\\
plotJsonData: string
\subsubsection{State Invariant}
DEFAULT=``./dataCenter/overalldata.json"\\
PATH = ``./dataCenter/plot"\\
SUFFIX = ``data.json"
\subsubsection{Environment Variables}
overalldata.json\\
plot1data.json\\
plot2data.json\\
plot3data.json\\
plot4data.json\\
plot5data.json\\
plot6data.json\\
plot7data.json\\
plot8data.json\\
plot9data.json\\
plot10data.json\\
plot11data.json\\
plot12data.json\\
plot13data.json\\
plot14data.json

\subsubsection{Assumptions}

Assume all the Json files are in the correct path.

\subsubsection{Access Routine Semantics}

\noindent Awake():
\begin{itemize}
\item transition: readFile(0)
\item output: None 
\item exception: None 
\end{itemize}
\noindent Start():
\begin{itemize}
\item transition: None
\item output: None 
\item exception: None 
\end{itemize}
\noindent readFile(value):
\begin{itemize}
\item transition: plotNumber:= value + 1,\\
(plotNumber=15) $\rightarrow$ (filePath:=DEFAULT) $\lor$ (plotNumber$\neq$15) $\rightarrow$ (filePath:=psx) WHERE p:=PATH, s:=plotNumber.ToString(), f:= SUFFIX,\\
plotJsonData:= File.ReadAllText(filePath),\\
JsonModelObj:= Newtonsoft.Json.JsonConvert.DeserializeObject<JsonModel>(plotJsonData),\\
        DataModelObj.RedPineData:=JsonModelObj.redPine;\\
        DataModelObj.OakData:=JsonModelObj.oak;\\
        DataModelObj.BeechData:=JsonModelObj.beech;\\
        DataModelObj.BirchData:=JsonModelObj.birch;\\
        DataModelObj.RedMapleData:=JsonModelObj.redMaple;\\
        DataModelObj.WhitePineData:=JsonModelObj.whitePine;\\
        DataModelObj.RedOakData:=JsonModelObj.redOak;\\
        DataModelObj.EnvData:=JsonModelObj.envData;\\
        
\item output: None 
\item exception: None 
\end{itemize}
\subsubsection{Local Functions}
None
%%%%%%%%%%%%%%%% reader ends

\newpage

%%%%%%%%%%%%%%%% writer starts %%%%%%%%%%%%%%%%%
\section{MIS of JSON File Writer Module (\mref{Controller2})}

\subsection{Module}
JsonFileWriter

\subsection{Uses}
JsonFileReader\\
NewDataInpputBoxController\\
System.Collections\\
System.Collections.Generic\\
UnityEngine\\
System.IO\\
UnityEngine.UI\\
\mref{Viwer21}\\
\mref{Viwer22}\\
\mref{Model12}\\
\mref{Model4}\\
\mref{Model5}\\
\mref{Model6}\\
\mref{Model7}\\
\mref{Model8}\\
\mref{Model9}\\
\mref{Model10}\\
\mref{Model11}\\

\subsection{Syntax}
\subsubsection{Exported Constants}
None
\subsubsection{Exported Access Programs}

\begin{center}
\begin{tabular}{|l|l|l|p{5cm}|}
\hline
\textbf{Name} & \textbf{In} & \textbf{Out} & \textbf{Exceptions} \\
\hline
updateData & string, $\mathbb{Z}$, string &  &  \\
\hline
getOldData& $\mathbb{Z}$ & & \\ 
\hline
changeData& $\mathbb{Z}$, string, string & & \\
\hline
writeAndSave & string, string  &   & \\
\hline
\end{tabular}
\end{center}

\subsection{Semantics}

\subsubsection{State Variables}
treeParamDisplay: TreeParamDisplay\\
envDataDisplay: EnvDataDisplay\\
dataModelObj: DataModel\\
jsonModelObj: JsonModel\\
plotNumber: $\mathbb{Z}$\\
filePath: string\\
plotJsonData: stri\\
\subsubsection{State Invariant}
DEFAULT=``./dataCenter/overalldata.json"\\
PATH = ``./dataCenter/plot"\\
SUFFIX = ``data.json"
\subsubsection{Environment Variables}
overalldata.json\\
plot1data.json\\
plot2data.json\\
plot3data.json\\
plot4data.json\\
plot5data.json\\
plot6data.json\\
plot7data.json\\
plot8data.json\\
plot9data.json\\
plot10data.json\\
plot11data.json\\
plot12data.json\\
plot13data.json\\
plot14data.json

\subsubsection{Assumptions}

Assume that all the JSON files are in the correct path.

\subsubsection{Access Routine Semantics}

\noindent findFilePath(value):
\begin{itemize}
\item transition: plotNumber:=value + 1,\\
(plotNumber=15) $\rightarrow$ (filePath:=DEFAULT) $\lor$ (plotNumber$\neq$15) $\rightarrow$ (filePath:=psx) WHERE p:=PATH, s:=plotNumber.ToString(), f:= SUFFIX\\
\item output: None
\item exception: None
\end{itemize}
\noindent getOldData(value):
\begin{itemize}
\item transition: JsonFileReader.readFile(value)
\item output: None
\item exception: None
\end{itemize}
\noindent changeData(value, tree, p):
\begin{itemize}
\item transition: treeData:=(tree $\rightarrow$ RedPineData $\lor$ OakData
$\lor$ BeechData
$\lor$ RedPineData
$\lor$ RedMapleData
$\lor$ WhitePineData
$\lor$ RedOakData),\\
treeData.p:=value, plotJsonData:= set of treeData\\
content:= serializeObject(plotJsonData);
\item output: None
\item exception: None
\end{itemize}
\noindent writeAndSave(content, filePath):
\begin{itemize}
\item transition: write(content)
\item output: None
\item exception: None
\end{itemize}

\subsubsection{Local Functions}
None

%%%%%%%%%%%%% writer ends %%%%%%%%%%%%
  \newpage

%%%%%%%%%%%% Pause manager starts %%%%%%%%%%%%
\section{MIS of Pause Manager Module (\mref{Controller3})}
\subsection{Module}

PauseManager

\subsection{Uses}
System.Collections\\
System.Collections.Generic\\
UnityEngine\\
UnityEngine.UI\\

\subsection{Syntax}

\subsubsection{Exported Constants}
None
\subsubsection{Exported Access Programs}

\begin{center}
\begin{tabular}{|l|l|l|p{5cm}|}
\hline
\textbf{Name} & \textbf{In} & \textbf{Out} & \textbf{Exceptions} \\
\hline
Start &  &  &  \\
\hline
Update & & & \\
\hline
\end{tabular}
\end{center}

\subsection{Semantics}

\subsubsection{State Variables}
isPaused: Boolean\\
pauseMessage: Text
\subsubsection{State Invariant}
PAUSE = ``Pause"\\
BLANK = ``"
\subsubsection{Environment Variables}

keyboard

\subsubsection{Assumptions}

None

\subsubsection{Access Routine Semantics}

\noindent Start():
\begin{itemize}
\item transition: pauseMessage.text:= BLANK
\item output: None 
\item exception: None
\end{itemize}
\noindent Update():
\begin{itemize}
\item transition: Input.GetKeyDown(KeyCode.P) $\rightarrow$ (	$\neg$isPaused $\rightarrow$ (Time.timescale:= 0, pauseMessage.text = PAUSE, isPaused:= $\neg$~isPaused) $\lor$ (isPaused $\rightarrow$  (Time.timescale:= 1, pauseMessage.text = BLANK, isPaused:= $\neg$~isPaused)))
\item output: None 
\item exception: None
\end{itemize}

\subsubsection{Local Functions}
None

%%%%%%%%%%%%%%%%%%%%%%%%% Pause manager ends %%%%%%%%%%

 \newpage

%%%%%%%%%%%%%%%%%%%%%%%%% Player Movement starts %%%%%%%%
\section{MIS of Player Movement Module (\mref{Controller4})}

\subsection{Module}
PlayerMovement

\subsection{Uses}

CharacterController\\
System.Collections\\
System.Collections.Generic\\
UnityEngine\\
Time\\
Vector3

\subsection{Syntax}

\subsubsection{Exported Constants}
None
\subsubsection{Exported Access Programs}

\begin{center}
\begin{tabular}{|l|l|l|p{5cm}|}
\hline
\textbf{Name} & \textbf{In} & \textbf{Out} & \textbf{Exceptions} \\
\hline
Update &  &  &  \\
\hline
\end{tabular}
\end{center}

\subsection{Semantics}

\subsubsection{State Variables}
speed: float\\
controller: CharacterController
transform: Transform
\subsubsection{Environment Variables}

mouse

\subsubsection{Assumptions}
Assume that users press the right keys.
\subsubsection{Access Routine Semantics}

\noindent Update():
\begin{itemize}
\item transition: x, z:= Input.GetAxis("Horizontal"),Input.GetAxis("Vertical"),\\
direction:= transform.right * x + Camera.main.transform.forward * z,\\
controller.Move(d,s,t): Vector3 $\times$ float $\times$ Time\\
\item output: None
\item exception: None 
\end{itemize}

\subsubsection{Local Functions}
None 
%%%%%%%%%%%%%%%%% Player Movement ends %%%%%%%%%%%%

\newpage

%%%%%%%%%%%%%%%%%%%%%%%%%%%%%%%%%%%%%%%%%%%%%
 \section{MIS of New Data Input Box Controller Module (\mref{Controller5})}
 
\subsection{Module}
NewDataInputBoxController

\subsection{Uses}
\mref{Controller2}

\subsection{Syntax}

\subsubsection{Exported Constants}
None
\subsubsection{Exported Access Programs}

\begin{center}
\begin{tabular}{|l|l|l|p{5cm}|}
\hline
\textbf{Name} & \textbf{In} & \textbf{Out} & \textbf{Exceptions} \\
\hline
isValid & string & & \\
\hline
storeData & string & &\\
\hline
\end{tabular}
\end{center}

\subsection{Semantics}

\subsubsection{State Variables}
number: string

\subsubsection{Environment Variables}
None
\subsubsection{Assumptions}

Assume that the contents of the string type input are all numbers 
\subsubsection{Access Routine Semantics}

\noindent isValid(number):
\begin{itemize}
\item transition: check if the input is valid or not
\item output: True if the input is valid, False otherwise
\item exception: None 
\end{itemize}
\noindent storeData(number):
\begin{itemize}
\item transition: call isValid(number) and pass the new data to the Update Data Button Module.
\item output: None
\item exception: None 
\end{itemize}

\subsubsection{Local Functions}

None
%%%%%%%%%%%%%%%%%%%%%%%%%%%%%%%%%%%%%%%%%%%%%%%%

\newpage

%%%%%%%%%%%%%%%%%%%%%%%%%%%%%%%%%%%%%%%%%%%%%
 \section{MIS of Start Button Controller Module (\mref{Controller6})}
 
\subsection{Module}
StartButtonController

\subsection{Uses}
System.Collections\\
System.Collections.Generic\\
UnityEngine\\
UnityEngine.SceneManagement\\
LoadSceneMode
\subsection{Syntax}

\subsubsection{Exported Constants}
newData: String
\subsubsection{Exported Access Programs}

\begin{center}
\begin{tabular}{|l|l|l|p{5cm}|}
\hline
\textbf{Name} & \textbf{In} & \textbf{Out} & \textbf{Exceptions} \\
\hline

Start & & & \\
\hline
Update & & & \\
\hline
goToForestScene & String, LoadSceneMode& &\\

\hline
\end{tabular}
\end{center}

\subsection{Semantics}
\subsubsection{State Variables}
None

\subsubsection{Environment Variables}
mouse\\
Forest

\subsubsection{Assumptions}

None
\subsubsection{Access Routine Semantics}

\noindent Start():
\begin{itemize}
\item transition: None 
\item output: None
\item exception: None
\end{itemize}
\noindent Update():
\begin{itemize}
\item transition: None 
\item output: None
\item exception: None
\end{itemize}
\noindent goToForestScene():
\begin{itemize}
\item transition: timescale := 1, load the Forest scene
\item output: None
\item exception: None
\end{itemize}


\subsubsection{Local Functions}
None
%%%%%%%%%%%%%%%%%%%%%%%%%%%%%%%%%%%%%%%

  \newpage
  
%%%%%%%%%%%%%%%%%%%%%%%%%%%%%%%%%%%%%%%%%%%%%%%
 \section{MIS of Instruction Button Controller Module (\mref{Controller7})}
 
\subsection{Module}

InstructionButtonController

\subsection{Uses}
None

\subsection{Syntax}

\subsubsection{Exported Constants}
None
\subsubsection{Exported Access Programs}

\begin{center}
\begin{tabular}{|l|l|l|p{5cm}|}
\hline
\textbf{Name} & \textbf{In} & \textbf{Out} & \textbf{Exceptions} \\
\hline
onClick & mouse click & &\\
\hline
setActive & Boolean &  &  \\
\hline
\end{tabular}
\end{center}

\subsection{Semantics}

\subsubsection{State Variables}
value: Boolean\\
active: Boolean
\subsubsection{Environment Variables}
Mouse\\
InstructionPage

\subsubsection{Assumptions}
None

\subsubsection{Access Routine Semantics}

\noindent onClick():
\begin{itemize}
\item transition: value:= $\neg$ value
\item output: None
\item exception: None
\end{itemize}

\noindent setActive(value):
\begin{itemize}
\item transition: active:= value
\item output: None
\item exception: None
\end{itemize}

\subsubsection{Local Functions}

None
%%%%%%%%%%%%%%%%%%%%%%%%%%%%%%%%%%%%%%%%%%%%%%%

\newpage

%%%%%%%%%%%%%%%%%%%%%%%%%%%%%%%%%%%%%%%%%%%%%%%
  \section{MIS of Update Data Button Controller Module (\mref{Controller8})}
\subsection{Module}

UpdateDataButtonController

\subsection{Uses}
JsonFileWriter
\subsection{Syntax}
\subsubsection{Exported Constants}
None
\subsubsection{Exported Access Programs}
\begin{center}
\begin{tabular}{|l|l|l|p{5cm}|}
\hline
\textbf{Name} & \textbf{In} & \textbf{Out} & \textbf{Exceptions} \\
\hline
onClick & mouse click & & \\
\hline
updateData &  string, $\mathbb{Z}$, string &  &  \\
\hline
\end{tabular}
\end{center}

\subsection{Semantics}

\subsubsection{State Variables}
value: Boolean\\
plot: $\mathbf{Z}$\\
tree: string

\subsubsection{Environment Variables}

Mouse\\
UpdateDataPage

\subsubsection{Assumptions}
None
\subsubsection{Access Routine Semantics}

\noindent onClick():
\begin{itemize}
\item transition: updateData(value, plot, tree)
\item output: None
\item exception: None
\end{itemize}

\noindent updateData(value, plot, tree):
\begin{itemize}
\item transition: active:= JsonFileWriter.write(value, plot, tree)
\item output: None
\item exception: None
\end{itemize}

\subsubsection{Local Functions}
None
%%%%%%%%%%%%%%%%%%%%%%%%%%%%%%%%%%%%%%%%%%%%%%%

\newpage  

%%%%%%%%%%%%%%%%%%%%%%%%%%%%%%%%%%%%%%%%%%%%%%%
\section{MIS of Contact Us Button Controller Module (\mref{Controller9})}
\subsection{Module}

ContactUsButtonController

\subsection{Uses}
None
\subsection{Syntax}
\subsubsection{Exported Constants}
None
\subsubsection{Exported Access Programs}
\begin{center}
\begin{tabular}{|l|l|l|p{5cm}|}
\hline
\textbf{Name} & \textbf{In} & \textbf{Out} & \textbf{Exceptions} \\
\hline
onClick & mouse click & & \\
\hline
setActive & Boolean &  &  \\
\hline
\end{tabular}
\end{center}

\subsection{Semantics}

\subsubsection{State Variables}
value: Boolean\\
active: Boolean
\subsubsection{Environment Variables}

Mouse\\
TeamInfoPage

\subsubsection{Assumptions}
None
\subsubsection{Access Routine Semantics}

\noindent onClick():
\begin{itemize}
\item transition: value:= $\neg$ value
\item output: None
\item exception: None
\end{itemize}

\noindent setActive(value):
\begin{itemize}
\item transition: active:= value
\item output: None
\item exception: None
\end{itemize}

\subsubsection{Local Functions}
None 
%%%%%%%%%%%%%%%%%%%%%%%%%%%%%%%%%%%%%%%%%%%%%%%

\newpage

%%%%%%%%%%%%%%%%%%%%%%%%%%%%%%%%%%%%%%%%%%%%%%%
\section{MIS of Quit Button Controller (\mref{Controller10})} 

\subsection{Module}

QuitButtonController

\subsection{Uses}
UnityEngine.UI (UI Library)
\subsection{Syntax}

\subsubsection{Exported Constants}
None
\subsubsection{Exported Access Programs}

\begin{center}
\begin{tabular}{|l|l|l|p{5cm}|}
\hline
\textbf{Name} & \textbf{In} & \textbf{Out} & \textbf{Exceptions} \\
\hline
onClick & mouse click &  &  \\
\hline
Start & & & \\
\hline
Update & & & \\
\hline
QuitSoftware & & terminate program & \\
\hline
\end{tabular}
\end{center}

\subsection{Semantics}

\subsubsection{State Variables}
None

\subsubsection{Environment Variables}
Mouse
\subsubsection{Assumptions}
None
\subsubsection{Access Routine Semantics}

\noindent Start():
\begin{itemize}
\item transition: None
\item output: None
\item exception: None
\end{itemize}

\noindent Update():
\begin{itemize}
\item transition: None
\item output: None
\item exception: None
\end{itemize}

\noindent QuitSoftware():
\begin{itemize}
\item transition: Application.Quit()
\item output: None
\item exception: None
\end{itemize}



\subsubsection{Local Functions}
None
%%%%%%%%%%%%%%%%%%%%%%%%%%%%%%%%%%%%%%%%%%%%%%%

\newpage

%%%%%%%%%%%%%%%%%%%%%%%%%%%%%%%%%%%%%%%%%%%%%%%
\section{MIS of Back Button Controller (\mref{Controller11})}  

\subsection{Module}
BackButtonController

\subsection{Uses}
UnityEngine.UI\\
UnityEngine.SceneManagement\\

\subsection{Syntax}
\subsubsection{Exported Constants}
None
\subsubsection{Exported Access Programs}

\begin{center}
\begin{tabular}{|l|l|l|p{5cm}|}
\hline
\textbf{Name} & \textbf{In} & \textbf{Out} & \textbf{Exceptions} \\
\hline
onClick & mouse click &  &  \\
\hline
Start & & & \\
\hline
Update & & & \\
\hline
Back & &  & \\
\hline
\end{tabular}
\end{center}

\subsection{Semantics}

\subsubsection{State Variables}
viewState\\
upperLevelPage

\subsubsection{Environment Variables}
Mouse
\subsubsection{Assumptions}
None
\subsubsection{Access Routine Semantics}

\noindent Start():
\begin{itemize}
\item transition: None
\item output: None
\item exception: None
\end{itemize}


\noindent Update():
\begin{itemize}
\item transition: None
\item output: None
\item exception: None
\end{itemize}


\noindent Back():
\begin{itemize}
\item transition: upperLevelPage $\mathit{\implies}$(viewState := upperLevelPage)
\item output: None
\item exception: None
\end{itemize}


\subsubsection{Local Functions}
None

%%%%%%%%%%%%%%%%%%%%%%%%%%%%%%%%%%%%%%%%%%%%%%%

\newpage

%%%%%%%%%%%%%%%%%%%%%%%%%%%%%%%%%%%%%%%%%%%%%%%
\section{MIS of Plot Selection Drop Down Controller (\mref{Controller12})} 

\subsection{Module}
PlotSelectionDropDownController

\subsection{Uses}
UnityEngine.UI\\
UnityEngine.SceneManagement\\

\subsection{Syntax}
\subsubsection{Exported Constants}
None
\subsubsection{Exported Access Programs}

\begin{center}
\begin{tabular}{| l | l | l | p{5cm}|}
\hline
\textbf{Name} & \textbf{In} & \textbf{Out} & \textbf{Exceptions} \\
\hline
onClick & mouse click &  &  \\
\hline
Start &&&\\
\hline
Update &&&\\
\hline
displayMenu &&&\\
\hline
extractTreeParam & s: int &  &  \\
\hline
\end{tabular}
\end{center}

\subsection{Semantics}

\subsubsection{State Variables}
isActive: Boolean\\
s1: String\\
s2: String\\
s3: String\\
s4: String\\
s5: String\\
curIndex: int

\subsubsection{Environment Variables}
Mouse\\
DataModelObj: The gameobject of the current script\\
EnvDisp: UI test that will be displayed in Unity
\subsubsection{Assumptions}
None
\subsubsection{Access Routine Semantics}

\noindent Start():
\begin{itemize}
\item transition: None
\item output: None
\item exception: None
\end{itemize}


\noindent Update():
\begin{itemize}
\item transition: None
\item output: None
\item exception: None
\end{itemize}

\noindent displayMenu():
\begin{itemize}
\item transition: isActive$\mathit{:= \neg }$ isActive
\item output: None
\item exception: None
\end{itemize}


\noindent extractTreeParam(s):
\begin{itemize}
\item transition: Get the mouse click, assign different values to s1,s2,s3,s4,s5 based on the value of curIndex
\item output: None
\item exception: None
\end{itemize}



\subsubsection{Local Functions}
None

%%%%%%%%%%%%%%%%%%%%%%%%%%%%%%%%%%%%%%%%%%%%%%%

\newpage

%%%%%%%%%%%%%%%%%%%%%%%%%%%%%%%%%%%%%%%%%%%%%%%
\section{MIS of Tree Type Selection Drop Down Controller(\mref{Controller13})} 

\subsection{Module}

TreeTypeSelectionDropDownController

\subsection{Uses}
UnityEngine.UI\\
UnityEngine.SceneManagement\\

\subsection{Syntax}

\subsubsection{Exported Constants}
None
\subsubsection{Exported Access Programs}

\begin{center}
\begin{tabular}{| l | l | l | p{5cm}|}
\hline
\textbf{Name} & \textbf{In} & \textbf{Out} & \textbf{Exceptions} \\
\hline
onClick & mouse click  &   & \\
\hline
Start &&&\\
\hline
Update &&&\\
\hline
displayMenu &&&\\
\hline
extractTreeParam & s: int &  &  \\
\hline
\end{tabular}
\end{center}

\subsection{Semantics}

\subsubsection{State Variables}
isActive: Boolean\\
curIndex: int\\
s1: String\\
s2: String\\
s3: String\\
s4: String\\
s5: String\\


\subsubsection{Environment Variables}
Mouse\\
DataModelObj: The gameobject of the current script\\
TreeParamDisp: UI test that will be displayed in Unity\\
dropdown: The drop down menu to select tree type
\subsubsection{Assumptions}
None
\subsubsection{Access Routine Semantics}

\noindent Start():
\begin{itemize}
\item transition: None 
\item output: None
\item exception: None
\end{itemize}

\noindent Update():
\begin{itemize}
\item transition: None 
\item output: None
\item exception: None
\end{itemize}

\noindent displayMenu():
\begin{itemize}
\item transition: isActive$\mathit{:= \neg }$ isActive
\item output: None
\item exception: None

\end{itemize}


\noindent extractTreeParam(s):
\begin{itemize}
\item transition: Get the mouse click, assign different values to s1,s2,s3,s4,s5 based on the value of curIndex
\item output: None
\item exception: None
\end{itemize}

\subsubsection{Local Functions}
None

%%%%%%%%%%%%%%%%%%%%%%%%%%%%%%%%%%%%%%%%%%%%%

\newpage

%%%%%%%%%%%%%%%%%%%%%%%%%%%%%%%%%%%%%%%%%%%%%%%
\section{MIS of Show Environmental Data Button Controller (\mref{Controller14})} 

\subsection{Module}

ShowEnvDataButtoController

\subsection{Uses}
UnityEngine.UI\\
UnityEngine.SceneManagement\\


\subsection{Syntax}

\subsubsection{Exported Constants}
None
\subsubsection{Exported Access Programs}

\begin{center}
\begin{tabular}{| l | l | l | p{5cm}|}
\hline
\textbf{Name} & \textbf{In} & \textbf{Out} & \textbf{Exceptions} \\
\hline
onClick & mouse click &  &  \\
\hline
Start &&&\\
\hline
Update &&&\\
\hline
EnvDataDispHandle &&&\\
\hline
\end{tabular}
\end{center}

\subsection{Semantics}

\subsubsection{State Variables}
displayEnvData: Boolean

\subsubsection{Environment Variables}
Mouse
\subsubsection{Assumptions}
None
\subsubsection{Access Routine Semantics}

\noindent Start():
\begin{itemize}
\item transition: None
\item output: None
\item exception: None
\end{itemize}


\noindent Update():
\begin{itemize}
\item transition: None
\item output: None
\item exception: None
\end{itemize}

\noindent EnvDataDispHandle():
\begin{itemize}
\item transition: displayEnvData $\mathit {:= \neg}$ displayEnvData
\item output: None
\item exception: None
\end{itemize}

\subsubsection{Local Functions}
None
%%%%%%%%%%%%%%%%%%%%%%%%%%%%%%%%%%%%%%%%%%%%%%%%%%%

\newpage

%%%%%%%%%%%%%%%%%%%%%%%%%%%%%%%%%%%%%%%%%%%%%%%%%%%
\section{MIS of Show Tree Parameter Button Controller(\mref{Controller15})}  

\subsection{Module}

ShowTreeParamButtonController

\subsection{Uses}
UnityEngine.UI\\
UnityEngine.SceneManagement\\

\subsection{Syntax}

\subsubsection{Exported Constants}
None
\subsubsection{Exported Access Programs}

\begin{center}
\begin{tabular}{| l | l | l | p{5cm}|}
\hline
\textbf{Name} & \textbf{In} & \textbf{Out} & \textbf{Exceptions} \\
\hline
onClick & mouse click &  &  \\
\hline
Start &&&\\
\hline
Update &&&\\
\hline
TreeParamDispHandle &&&\\
\hline
\end{tabular}
\end{center}

\subsection{Semantics}

\subsubsection{State Variables}
isActive: Boolean

\subsubsection{Environment Variables}
Mouse
\subsubsection{Assumptions}
None
\subsubsection{Access Routine Semantics}

\noindent Start():
\begin{itemize}
\item transition: None
\item output: None
\item exception: None
\end{itemize}

\noindent Update():
\begin{itemize}
\item transition: None
\item output: None
\item exception: None
\end{itemize}

\noindent TreeParamDispHandle():
\begin{itemize}
\item transition: isActive $\mathit{:= \neg}$ isActive
\item output: None
\item exception: None
\end{itemize}
\subsubsection{Local Functions}
None

%%%%%%%%%%%%%%%%%%%%%%%%%%%%%%%%%%%%%%%%%%

\newpage

%%%%%%%%%%%%%%%%%%%%%%%%%%%%%%%%%%%%%%%%%%%%

\section{MIS of Environmental Selection Button Controller(\mref{Controller16})} 

\subsection{Module}
EnvDataSelectionButtonController


\subsection{Uses}
UnityEngine.UI\\
UnityEngine.SceneManagement\\

\subsection{Syntax}

\subsubsection{Exported Constants}
None
\subsubsection{Exported Access Programs}

\begin{center}
\begin{tabular}{| l | l | l | p{5cm}|}
\hline
\textbf{Name} & \textbf{In} & \textbf{Out} & \textbf{Exceptions} \\
\hline
onClick & mouse click &  &  \\
\hline
Start &&&\\
\hline
Update &&&\\
\hline
displayEnvSel &&&\\
\hline
\end{tabular}
\end{center}

\subsection{Semantics}

\subsubsection{State Variables}
isActive: Boolean

\subsubsection{Environment Variables}
Mouse
\subsubsection{Assumptions}
None
\subsubsection{Access Routine Semantics}

\noindent Start():
\begin{itemize}
\item transition: None
\item output: None
\item exception: None
\end{itemize}

\noindent Update():
\begin{itemize}
\item transition: None
\item output: None
\item exception: None
\end{itemize}

\noindent displayEnvSel():
\begin{itemize}
\item transition: isActive $\mathit{:= \neg}$ isActive
\item output: None
\item exception: None
\end{itemize}
\subsubsection{Local Functions}
None

%%%%%%%%%%%%%%%%%%%%%%%%%%%%%%%%%%%%%%%%%%%%%%%%%

\newpage

%%%%%%%%%%%%%%%%%%%%%%%%%%%%%%%%%%%%%%%%%%%%%%%%%
\section{MIS of Data Type Selection Buttons Controller(\mref{Controller17})}  

\subsection{Module}
DataTypeSelectionButtonsController

\subsection{Uses}
UnityEngine.UI\\
UnityEngine.SceneManagement\\

\subsection{Syntax}

\subsubsection{Exported Constants}
None
\subsubsection{Exported Access Programs}

\begin{center}
\begin{tabular}{| l | l | l | p{5cm}|}
\hline
\textbf{Name} & \textbf{In} & \textbf{Out} & \textbf{Exceptions} \\
\hline
onClick & mouse click &  &  \\
\hline
Start &&&\\
\hline
Update &&&\\
\hline
displayDataTypeSel &&&\\
\hline
\end{tabular}
\end{center}

\subsection{Semantics}

\subsubsection{State Variables}
isActive: Boolean

\subsubsection{Environment Variables}
Mouse
\subsubsection{Assumptions}
None
\subsubsection{Access Routine Semantics}

\noindent Start():
\begin{itemize}
\item transition: None
\item output: None
\item exception: None
\end{itemize}

\noindent Update():
\begin{itemize}
\item transition: None
\item output: None
\item exception: None
\end{itemize}

\noindent displayDataTypeSel():
\begin{itemize}
\item transition: isActive $\mathit{:= \neg}$ isActive
\item output: None
\item exception: None
\end{itemize}

\subsubsection{Local Functions}
None

%%%%%%%%%%%%%%%%%%%%%%%%%%%%%%%%%%%%%%%%%%%%%%%%%%%

\newpage

%%%%%%%%%%%%%%%%%%%%%%%%%%%%%%%%%%%%%%%%%%%%%%%%%%%%

\section{MIS of Save Button Controller(\mref{Controller18})}  

\subsection{Module}

SaveButtonController

\subsection{Uses}
UnityEngine.UI\\
UnityEngine.SceneManagement\\
\mref{Controller2}

\subsection{Syntax}
\subsubsection{Exported Constants}
None
\subsubsection{Exported Access Programs}

\begin{center}
\begin{tabular}{| l | l | l | p{5cm}|}
\hline
\textbf{Name} & \textbf{In} & \textbf{Out} & \textbf{Exceptions} \\
\hline
onClick & mouse click &  &  \\
\hline
Start &&&\\
\hline
Update &&&\\
\hline
Save &originalData &&\\
\hline
&updatedData&&\\
\hline
\end{tabular}
\end{center}

\subsection{Semantics}

\subsubsection{State Variables}
originalData: float\\
updatedData: float
\subsubsection{Environment Variables}
None
\subsubsection{Assumptions}
None
\subsubsection{Access Routine Semantics}

\noindent Start():
\begin{itemize}
\item transition: None
\item output: None
\item exception: None
\end{itemize}

\noindent Update():
\begin{itemize}
\item transition: None
\item output: None
\item exception: None
\end{itemize}

\noindent Save():
\begin{itemize}
\item transition: originalData := updatedData
\item output: None
\item exception: None
\end{itemize}


\subsubsection{Local Functions}
None

%%%%%%%%%%%%%%%%%%%%%%%%%%%%%%%%%%%%%%%%%%%%%%%%

\newpage

\bibliographystyle {plainnat}
\bibliography {References}

\newpage

\section{Appendix} \label{Appendix}

\end{document}